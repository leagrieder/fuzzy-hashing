\include{Introduction.tex}

\section{Biometric Setting}
\label{sec:bio_setting}

In this section, we explore the representation and processing of biometric data, specifically focusing on finger vein images. These images are converted into bitstrings to enhance processing efficiency. Each bitstring represents the presence or absence of vein pixels, forming the basis for biometric templates. The similarity between two biometric captures is quantified using Hamming weights and distances, enabling the calculation of a similarity score. This score is crucial for authentication and identification processes, ensuring accurate and reliable matching of biometric data.

\subsection{Biometric Data and Similarity Scores}
\label{Bio_data_sim_scores}
We begin by delving into the representation of the biometric data. Finger images, designated as biometric templates and formated as \(240 \times 376\), result in \(n\) = \(90'240\) pixels per image. To enhance processing efficiency, these templates are converted into vectors, departing from their original 2-dimension image structure. In the biometric context, each finger serves as a biometric subject, with a corresponding biometric capture represented as a bitstring \(X\) of length \(n\). This bitstring encapsulates the specific vein pixel information extracted from the finger image, while the biometric template serves as a reference model derived from these images. \\

Each of the \(n\) bits of the biometric capture is designated as \(X_1, \ldots, X_n\), with \(X_i\) set to \(1\) if the \(i\)-th bit corresponds to a vein and \(0\) otherwise. We define the random variable \(\left( \frac{\text{HW}(X)}{n} \right)\), as the Hamming Weight of a feature vector\footnote{The terms "feature vector" and "bitstring" are used interchangeably throughout this report to refer to the vein pixel information extracted from the finger image.} \(X\) divided by the total number of bits in the vector. In the case where \(i\) is a uniformly distributed random index and \(X\) is randomly chosen, we define the value \(p\)\footnote{Throughout this report, experimentally derived variables are denoted with the superscript "obs" (e.g., \(\text{var}^{\text{obs}}\)), while theoretical variables are presented without any superscript.} as the mean of this random variable.

\begin{equation} \label{eq:p}
    \begin{aligned}
        p = \mathbb{E}\left( \frac{\text{HW}(X)}{n} \right) = Pr\left[X_i = 1\right]
    \end{aligned}
\end{equation}

\newpage
The experimental results for the mean (\(p\)), along with the standard deviation (\(\sigma_p\))\footnote{The notation, \( \sigma_{p}\) denotes the theoretical standard deviation, and \(\sigma^2_p \) represents the theoretical variance of the random variable \(\left( \frac{\text{HW}(X)}{n} \right)\). Here $p$ refers to the mean of this random variable. These notations will be used consistently throughout our report.} and variance (\(\sigma_{p}²\)) of the random variable \(\left( \frac{\text{HW}(X)}{n} \right)\) across all images will be detailed in Section~\ref{sec:p}, and are quantified as follows:

\begin{equation} \label{eq:proba1}
    p^{\text{obs}} = \mathbb{E}\left( \frac{\text{HW}(X)}{n} \right) = 3.51\%
\end{equation}


\begin{equation} \label{eq:proba2}
    \sigma_{\text{p}}^{\text{obs}} = \sqrt{\mathbb{E} \left[ \left( \frac{\text{HW}(X)}{n} - p^{\text{obs}} \right)^2 \right]} \approx 5.02 \times 10^{-3}
\end{equation}

\begin{equation} \label{eq:proba3}
    (\sigma_{\text{p}}^{\text{obs}})² = \mathbb{V}\left( \frac{\text{HW}(X)}{n} \right)  \approx 2.52 \times 10^{-5}
\end{equation}\\

In biometric authentication and identification, the uniqueness of each biometric capture is encoded in its bits. The objective revolves around discerning the similarity between two biometric captures to verify or identify two individuals. Therefore, the scoring mechansim plays a crucial role in quantatively determining the similarity between biometric captures. This similarity score holds significance in verifying the identity of an individual (authentication) or identifying potential matches in a database (identification). A higher score indicates a greater ressemblance between captures, while a lower score suggests less similarity. This scoring mechanism is indispensable for ensuring the accuracy and reliability of biometric systems. The score of (\(X\), \(Y\)) is computed as

\begin{equation} \label{eq:score}
    \begin{aligned}
        Score(X, Y) &= \frac{HW(X \land Y)}{HW(X) + HW(Y)}\\
        &= \frac{1}{2}-\frac{1}{2}\frac{d_H(X, Y)}{HW(X) + HW(Y)}
    \end{aligned}
\end{equation}

where \(HW\) denotes the \hyperref[def:Hamming Weight]{Hamming Weight} and \(d_H\) the \hyperref[def:Hamming Distance]{Hamming Distance}.\\ 
\newpage
In conjunction with the scoring mechanism, Miura matching emerges as a specialized technique for comparing biometric samples. This method entails determining an optimal offset translation, denoted as \(d_X\) and \(d_Y\), between two biometric samples to align their features for comparison, thereby compensating for differences in positioning or orientation. The alignment process maximizes the similarity score, defined as:

\[Score = \text{Score}(d_X \cdot X, d_Y \cdot Y)\]

Once the optimal offsets, \(d_X\) and \(d_Y\), are determined, they are applied to the original samples for alignment. This results in the calculated aligned positions, denoted as \(\bar{X}\) and \(\bar{Y}\), such that:

\[\bar{X} = d_X \cdot X\]
\[\bar{Y} = d_Y \cdot Y\]\\

The probability of the \(i\)-th pixel of two captures not being the same after applying the optimal offset translations depends on the distribution of \((\bar{X}, \bar{Y})\), and is denoted as \(\delta\). We define the random variable \(\left( \frac{d_H(\bar{X}, \bar{Y})}{n} \right)\) as the normalized Hamming distance between two feature vectors. The parameter \(\delta\) is defined as the mean of this random variable, and we additionally define the standard deviation (\(\sigma_{\delta}\))\footnote{The notation, \(\sigma_{\delta}\) denotes the theoretical standard deviation, and \(\sigma^2_{\delta} \) represents the theoretical variance of the random variable \(\left( \frac{d_H(\bar{X}, \bar{Y})}{n} \right)\). Here $\delta$ refers to the mean of this random variable. These notations will be used consistently throughout our report.} and the variance (\(\sigma^2_{\delta}\)) of this random variable, as shown in the following figures (where \(i\) is a uniformly distributed random index).


\begin{equation} \label{eq:delta}
    \begin{aligned}
        \delta = \mathbb{E}\left( \frac{d_H(\bar{X}, \bar{Y})}{n} \right) = Pr[\bar{X}_i \neq \bar{Y}_i]
    \end{aligned}
\end{equation}

\begin{equation}
    \begin{aligned}
        \sigma_{\delta} = {\sigma}\left( \frac{d_H(\bar{X}, \bar{Y})}{n} \right)
    \end{aligned}
\end{equation}

\begin{equation}
    \begin{aligned}
        (\sigma_{\delta})² = \mathbb{V}\left( \frac{d_H(\bar{X}, \bar{Y})}{n} \right)
    \end{aligned}
\end{equation}\\


Distinguishing between the types of distributions associated with the two captures is crucial. When both captures originate from the same biometric subject, it is referred to as \(\delta_{\text{same}}\). Conversely, if the captures are from different subjects, it is labeled as \(\delta_{\text{diff}}\). Additionally, when \(\bar{X}\) and \(\bar{Y}\) consist of \(2n\) independent random bits with an expected value of \(p\) and no optimal offset is applied, it is classified as \(\delta_{\text{indep}}\). In this scenario, \(\delta_{\text{indep}}^{\text{obs}} = 2p^{\text{obs}}(1-p^{\text{obs}}) = 6.78\%\). As will be detailed in Section~\ref{sec:delta}, the experimental results produce the following figures:

\begin{table}[H]
    \centering
    \renewcommand{\arraystretch}{1.25}
    \begin{tabular}{|c|c|c|c|}
        \hline
        & \text{\(\delta^{\text{obs}}\)} & \text{\(({\sigma^{\text{obs}}_{\delta}})²\)} & \text{\(\sigma_{\delta}^{\text{obs}}\)} \\
        \hline
        \text{Same Biometric Subjects} & 4.55\% & \(1.06 \times 10^{-4}\) & \(1.03 \times 10^{-2}\) \\
        \hline
        \text{Different Biometric Subjects} & 5.99\% & \(4.58 \times 10^{-5}\) & \(6.76 \times 10^{-3}\) \\
        \hline
    \end{tabular}
    \caption{Comparison of Distributions: Mean, Standard Deviation, and Variance of the random variable \(\left( \frac{d_H(\bar{X}, \bar{Y})}{n} \right)\) when both captures originate from the same and from different biometric subjects}
\end{table}

Utilizing the formulas for \(p\) (\ref{eq:p}) and $\delta$ (\ref{eq:delta}), the joint distribution for (\(\bar{X}_j\), \(\bar{Y}_j\)) across \(j\) random instances is derived:

\begin{table}[H]
    \centering
    \renewcommand{\arraystretch}{1.5}
    \begin{tabular}{|c|c|c|}
        \hline
        & $\bar{Y}_i = 0$ & $\bar{Y}_i = 1$\\
        \hline
        $\bar{X}_i = 0$ & $1 - p - \frac{\delta}{2}$ & $\frac{\delta}{2}$\\
        \hline
        $\bar{X}_i = 1$ & $\frac{\delta}{2}$ & $p - \frac{\delta}{2}$\\
        \hline
    \end{tabular}
    \caption{Joint Distribution of ($\bar{X}_j$, $\bar{Y}_j$) for Random Instances}
    \label{tab:joint_distribution}
\end{table}

\subsection{Experimental Derivation of the Probability \(p\)}
\label{sec:p}
In order to derive the probability that a distributed random pixel is a vein (\(p\)), a dataset of 20 unique individuals was utilized. Each individual contributed images of both right and left index/middle fingers, with 5 trials per finger, captured using 2 different cameras. This methodology resulted in a comprehensive dataset of 800 images. Following the approach in Section~\ref{sec:extraction-pipeline}, Simon's optimal pipeline was implemented to extract the feature vectors from each image in the dataset (refer to Figure~\ref{pipeline_simon}). Subsequently, statistical analysis was performed on these feature vectors to gain insights into vein patterns, building on the preliminary work by Burcu. It's important to note that the calculation of the probability \(p\) does not take into account any postalignment method yet.

In order to derive the mean of \(\left( \frac{HW(X)}{n} \right)\), i.e. \(p\), we have implemented an algorithm which processes the dataset's feature vectors, performing two main functions:
\begin{enumerate}
    \item For each pixel position, it updates a \textit{veins[i]} array, which accumulates detections of veins across all images for each camera \(i={1,2}\).
    \item It maintains a count of the number of images analyzed per camera \(i={1,2}\) in an \textit{image\_count[i]} array.
\end{enumerate}

Following the processing of all feature vectors, the algorithm computes the probability of a pixel being part of a vein by dividing the aggregated vein detections by the total number of images analyzed, across both cameras. Visual representations were generated to illustrate the distribution of detected vein pixels for each camera individually and for the combined data from both cameras.

\begin{enumerate}
    \item \textbf{Camera 1 and Camera 2 distribution}: The following histograms showcase the frequency distribution of vein pixels detected in images captured by camera 1 and 2 individually, with a Gaussian fit overlaid to highlight the data's normal distribution trend.

    \begin{figure}[H]
        \centering
        \includegraphics[width=1\linewidth]{latex-img/distribution_veins_cam1.png}
        \caption{Vein Pixel Distribution Analysis Using Camera 1: Histogram Representation with Gaussian Fit Overlay with \(X_i \sim \mathcal{N}(3301.28, 216433.17)\)}
        \label{distribution_veins_cam1}
    \end{figure}

    \begin{figure}[H]
        \centering
        \includegraphics[width=1\linewidth]{latex-img/distribution_veins_cam2.png}
        \caption{Vein Pixel Distribution Analysis Using Camera 2: Histogram Representation with Gaussian Fit Overlay \(X_i \sim \mathcal{N}(3041.04, 159577.13)\)}
        \label{distribution_veins_cam2}
    \end{figure}
 
    \newpage
    \item \textbf{Combined Cameras Distribution}: To understand the aggregate behavior of vein detection across both cameras, the data was merged, generating a comprehensive histogram with a Gaussian fit.

    \begin{figure}[H]
        \centering
        \includegraphics[width=1\linewidth]{latex-img/distribution_veins_bothcams.png}
        \caption{Aggregate Vein Pixel Distribution Analysis: Combined Histogram and Gaussian Fit from Both Cameras \(X_i \sim \mathcal{N}(3171.16, 204935.79)\)}
        \label{distribution_veins_bothcams}
    \end{figure}
\end{enumerate}


Concluding the analysis of the vein pixel distribution across different camera captures and their aggregate dataset, the average probability \(p\) that a given pixel corresponds to a vein was computed. This involved summing up all vein-identifying pixels within the datasets for Camera 1, Camera 2, and the combined dataset. Subsequently, this sum was divided by the total number of pixels processed, multiplying the count of images by the total pixels per image, to ascertain the average likelihood \(p\) that any randomly selected pixel is part of a vein pattern.

The findings from this analysis revealed the following probabilities:
\begin{enumerate}
    \item For camera 1, the probability \(p^{\text{obs}} = \mathbb{E}\left( \frac{\text{HW}(X)}{n} \right)\) was calculated to be \(0.03658\), indicating an approximate \(3.66\)\% chance that any given pixel in images from Camera 1 represents a vein.

    \item For camera 2, the probability \(p^{\text{obs}} = \mathbb{E}\left( \frac{\text{HW}(X)}{n} \right)\) was slightly lower at \(0.03369\), translating to an approximate \(3.37\)\% chance for vein representation in its images.

    \item When considering the datasets from both cameras combined, the probability \(p^{\text{obs}} = \mathbb{E}\left( \frac{\text{HW}(X)}{n} \right)\) averaged out to \(0.03514\), suggesting an approximate \(3.51\)\% likelihood of a pixel depicting a vein across the entire dataset. Moreover, the standard deviation (\( \sigma^{\text{obs}}_p \)) and the variance (\((\sigma^{\text{obs}}_p)²\)) of the random variable \(\left( \frac{\text{HW}(X)}{n} \right)\) provide insights into the the random variable's spread and consistency. The observed variance is low, at approximately \( 2.52 \times 10^{-5} \), indicating minimal fluctuation in the random variables' values among different feature vectors. This suggests that the feature extraction method is highly reliable and consistent across different images. Furthermore, the standard deviation, calculated as approximately \( 5.02 \times 10^{-3} \), reinforces the idea of minimal dispersion around the mean probability (\(p\)) of detecting a vein. 
    
\end{enumerate}

\subsection{Experimental Derivation of the Probabilities \(\delta_{\text{same}}, \delta_{\text{diff}}, \delta_{\text{indep}}\)}
\label{sec:delta}

To assess the probability that the i-th pixel of two captures diverges after applying the optimal offset translations \( d_X \) and \( d_Y \) (Equation \ref{eq:delta}), we undertake additional experimental analysis. This investigation continues to utilize the same dataset comprising 20 distinct individuals. Here, the post-alignment process, specifically Miura Matching, is included in our examination as the calculation formula integrates this step. As detailed in Section~\ref{Bio_data_sim_scores}, our objective is to evaluate three distinct delta values: \(\delta_{\text{same}}\), \(\delta_{\text{diff}}\), and  \(\delta_{\text{indep}}\).\\

The computation of \( \delta_{\text{indep}} \) is relatively straightforward, having determined \( p^{\text{obs}} \). In order to calculate the mean, the standard deviation, and the variance of the random variable \(\left( \frac{d_H(\bar{X}, \bar{Y})}{n} \right)\) when \(X\) and \(Y\) are made up of \(2n\) independent random bits of expected value \(p\) and without applying the optimal offset to get \(\bar{X}\) and \(\bar{Y}\), we can proceed as follows:

\[ \delta^{\text{obs}}_{\text{indep}} = 2p^{\text{obs}}(1 - p^{\text{obs}}) = 2 \times 0.03514 \times (1 - 0.03514) \approx 6.78\% \]

Next, the standard deviation and the variance of the normalized Hamming distance are given as:

\[ {\sigma^{\text{obs}}_{\delta_{\text{indep}}}} = \sqrt{\mathbb{E} \left[ \left( \frac{d_H(X, Y)}{n} - \delta_{\text{indep}} \right)^2 \right]} = \sqrt{\frac{4p^{\text{obs}}(1 - p^{\text{obs}})(1 - 2p^{\text{obs}}(1 - p^{\text{obs}}))}{n}} \approx 1.18 \times 10^{-3} \]

\[ ({\sigma^{\text{obs}}_{\delta_{\text{indep}}}})² = \mathbb{V}\left( \frac{d_H(X, Y)}{n} \right) = \frac{4p^{\text{obs}}(1 - p^{\text{obs}})(1 - 2p^{\text{obs}}(1 - p^{\text{obs}}))}{n} \approx 1.40 \times 10^{-6} \]\\

To accurately evaluate \( \delta_{\text{same}} \) and \( \delta_{\text{diff}} \), a meticulous procedure was executed on our dataset, which involves 20 unique individuals, each having multiple biometric captures. Below is an outline of the methodology applied:

\begin{enumerate}
    \item \textbf{Pairwise Comparison}: For each individual, we compared every biometric capture to every other capture within the dataset, applying the Hamming distance metric to compute the normalized distances.
    \newpage
    \item \textbf{Statistical Analysis}: We organized the resulting distances into two distinct categories:
    \begin{itemize}
        \item Intra-individual distances, where captures from the same individual (same finger images) were compared, forming the first group.
        \item Inter-individual distances, comprising comparisons between biometric captures from different individuals, forming the second group.
    \end{itemize}
    \item \textbf{Probability Computation}:
    \begin{itemize}
        \item For \( \delta_{\text{same}} \), we calculated the average normalized distances from intra-individual comparisons. This analysis provided the following results, quantifying the observed differences between captures from the same individual after alignment:

        \[ \delta_{\text{same}}^{\text{obs}} \approx 4.55\% \]
        \[ \sigma^{\text{obs}}_{\delta_{\text{same}}} \approx 1.03 \times 10^{-2} \]
        \[ (\sigma^{\text{obs}}_{\delta_{\text{same}}})² \approx 1.06 \times 10^{-4} \]

        \begin{figure}[H]
            \centering
            \includegraphics[width=0.7\linewidth]{latex-img/delta_same.png}
            \caption{Frequency Distribution of Normalized Hamming Distances for Identical Biometric Samples with Alignment \(\delta_{\text{same}} \sim \mathcal{N}(0.05, 1.06 \times 10^{-4})\)}
            \label{delta_same}
        \end{figure}
        
        \newpage
        \item For \( \delta_{\text{diff}} \), we performed a similar averaging of the normalized distances from the inter-individual comparisons, which furnished us with the following results for observing a difference between captures from the different individuals after alignment:

        \[ \delta_{\text{diff}}^{\text{obs}} \approx 5.99\% \]
        \[ \sigma^{\text{obs}}_{\delta_{\text{diff}}} \approx 6.76 \times 10^{-3} \]
        \[ (\sigma^{\text{obs}}_{\delta_{\text{diff}}})² \approx 4.58 \times 10^{-5} \]

        \begin{figure}[H]
            \centering
            \includegraphics[width=0.7\linewidth]{latex-img/delta_diff.png}
            \caption{Frequency Distribution of Normalized Hamming Distances for Different Biometric Samples with Alignment \(\delta_{\text{diff}} \sim \mathcal{N}(0.06, 4.58 \times 10^{-5})\)}
            \label{delta_diff}
        \end{figure}
    \end{itemize}
\end{enumerate}

The joint probability distribution presented in Table \ref{tab:joint_distribution} represents the relationship between the variables \(\bar{X}_i\) and \(\bar{Y}_i\), which compare the i-th bit of two biometric samples. The parameter \(p\) denotes the probability of a bit being 1, and by complement, \(1-p\) signifies the probability of a bit being 0. The parameter \(\delta\) encapsulates the probability that the two bits are not identical.

In Table \ref{tab:joint_distribution}, we explain each entry:

\begin{itemize}
    \item By utilizing the previously determined value of $\delta$, we can infer the probabilities \(Pr[\bar{X}_i = 1, \bar{Y}_i = 0]\) and \(Pr[\bar{X}_i = 0, \bar{Y}_i = 1]\) as follows: 
    \begin{equation*}
        \begin{aligned}
            \delta = Pr[\bar{X}_i \neq \bar{Y}_i] = Pr[\bar{X}_i = 1, \bar{Y}_i = 0] + Pr[\bar{X}_i = 0, \bar{Y}_i = 1]
        \end{aligned}
    \end{equation*}
    Hence we have:
    \begin{equation*}
        \begin{aligned}
            Pr[\bar{X}_i = 1, \bar{Y}_i = 0] = Pr[\bar{X}_i = 0, \bar{Y}_i = 1] = \frac{\delta}{2}
        \end{aligned}
    \end{equation*}
    \item To determine the probability\(Pr[\bar{X}_i = 0, \bar{Y}_i = 0]\), we consider the following relationships and utilize the law of total probability:
    \begin{equation*}
        \begin{aligned}
            Pr[\bar{X}_i = 0] = 1-p = Pr[\bar{X}_i = 0, \bar{Y}_i = 0] + Pr[\bar{X}_i = 0, \bar{Y}_i = 1]
        \end{aligned}
    \end{equation*} 
    \begin{equation*}
        \begin{aligned}
            Pr[\bar{Y}_i = 0] = 1-p = Pr[\bar{X}_i = 0, \bar{Y}_i = 0] + Pr[\bar{X}_i = 1, \bar{Y}_i = 0]
        \end{aligned}
    \end{equation*} 
    
    By solving these equations, we can derive the probability of interest, \\\(Pr[\bar{X}_i = 0, \bar{Y}_i = 0]\), and hence infer the previously obtained values for \\\(Pr[\bar{X}_i = 1, \bar{Y}_i = 0]\) and \(Pr[\bar{X}_i = 0, \bar{Y}_i = 1]\):
    \[ Pr[\bar{X}_i = 0, \bar{Y}_i = 0] = 1-p - \frac{\delta}{2} \] 

    \item The probability \(Pr[\bar{X}_i = 1, \bar{Y}_i = 1]\) is derived by 
    \[Pr[\bar{X}_i = 1, \bar{Y}_i = 1] = 1 - Pr[\bar{X}_i = 0, \bar{Y}_i = 0] = p-\frac{\delta}{2}\]

\end{itemize}

Utilizing \(\delta_{\text{same}}^{\text{obs}}\), \(\delta_{\text{diff}}^{\text{obs}}\) and \(\delta_{\text{indep}}^{\text{obs}}\), we can determine the corresponding probabilities presented in Table \ref{tab:joint_distribution}. This gives us the following figures:


\begin{table}[H]
    \centering
    \renewcommand{\arraystretch}{1.5}
    \begin{tabular}{|c|c|c|}
        \hline
        & $Y_i = 0$ & $Y_i = 1$\\
        \hline
        $X_i = 0$ & $(1-p)² = 93.1\% $ & $p(1-p) = 3.39\%$\\
        \hline
        $X_i = 1$ & $p(1-p) = 3.39\% $ & $p² = 0.12\%$\\
        \hline
    \end{tabular}
    \caption{Joint Distribution of ($X_j$, $Y_j$) for \(X\) and \(Y\) made up of \(2n\) independent random bits of expected value \(p\)}
    \label{tab:joint_distribution_deltaindep}
\end{table}


\begin{table}[H]
    \centering
    \renewcommand{\arraystretch}{1.5}
    \begin{tabular}{|c|c|c|}
        \hline
        & $\bar{Y}_i = 0$ & $\bar{Y}_i = 1$\\
        \hline
        $\bar{X}_i = 0$ & $1 - p - \frac{\delta_{\text{same}}^{\text{obs}}}{2} = 94.21\% $ & $\frac{\delta_{\text{same}}^{\text{obs}}}{2} = 2.27\%$\\
        \hline
        $\bar{X}_i = 1$ & $\frac{\delta_{\text{same}}^{\text{obs}}}{2} = 2.27\%$ & $p - \frac{\delta_{\text{same}}^{\text{obs}}}{2} = 1.24\%$\\
        \hline
    \end{tabular}
    \caption{Joint Distribution of ($\bar{X}_j$, $\bar{Y}_j$) for \(X\) and \(Y\)  Originating from the Same Biometric Subject}
    \label{tab:joint_distribution_deltasame}
\end{table}

\begin{table}[H]
    \centering
    \renewcommand{\arraystretch}{1.5}
    \begin{tabular}{|c|c|c|}
        \hline
        & $\bar{Y}_i = 0$ & $\bar{Y}_i = 1$\\
        \hline
        $\bar{X}_i = 0$ & $1 - p - \frac{\delta_{\text{diff}}^{\text{obs}}}{2} = 93.49\% $ & $\frac{\delta_{\text{diff}}^{\text{obs}}}{2} = 2.99\%$\\
        \hline
        $\bar{X}_i = 1$ & $\frac{\delta_{\text{diff}}^{\text{obs}}}{2} = 2.99\%$ & $p - \frac{\delta_{\text{diff}}^{\text{obs}}}{2} = 0.52\%$\\
        \hline
    \end{tabular}
    \caption{Joint Distribution of ($\bar{X}_j$, $\bar{Y}_j$) for \(X\) and \(Y\) Originating from Different Biometric Subjects}
    \label{tab:joint_distribution_deltadiff}
\end{table}








