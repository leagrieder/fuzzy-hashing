\documentclass[12pt, a4paper, openany]{article}
\usepackage{fontspec}
\usepackage{chngcntr}
\usepackage{makecell}  % Include the makecell package



% LANGUAGE
\usepackage[english]{babel}
% \usepackage{enumitem}  % Enumerate improved
\usepackage[shortlabels]{enumitem}

% MATH / Others
\usepackage{amsmath, amssymb}  % Math symbols
\usepackage{physics}  % \norm and \abs
\usepackage{esvect, cancel}  % Misc., vectors, strikethrough
\usepackage{mhchem}  % Chemistry
\usepackage{siunitx}  % Units SI
\usepackage{amsmath}
\usepackage{algorithm}
\usepackage{algpseudocode}

% GEOMETRY
\usepackage[
    paper=a4paper,
    top=3cm,
    left=3cm,
    bottom=3cm,
    right=3cm,
    headheight=15pt,
    headsep=12pt,
]{geometry}
\usepackage{parskip}  % Reformat paragraphs, no indent first line
\usepackage{enumitem}  % Enumerate improved
\usepackage{scrextend}  % Indent text with addmargin environment
\usepackage{graphicx}  % Include graphics
\graphicspath{{latex-img}}
\usepackage{caption}  % Caption without figures
\usepackage{float}
\usepackage{multirow}
\usepackage{multicol}
\usepackage{subcaption}


% For references
\usepackage[backend=biber, style=numeric, citestyle=numeric]{biblatex}
\addbibresource{parts/References.bib}

\DeclareFieldFormat{labelnumber}{\textcolor{blue}{#1}} % Predefined color

\usepackage{booktabs} % For formal tables

\usepackage{tikz}
\usetikzlibrary{arrows}
\usetikzlibrary{shapes}
\newcommand{\mymk}[1]{%
  \tikz[baseline=(char.base)]\node[anchor=south west, draw,rectangle, rounded corners, inner sep=2pt, minimum size=7mm,
    text height=2mm](char){\ensuremath{#1}} ;}

\newcommand*\circled[1]{\tikz[baseline=(char.base)]{
            \node[shape=circle,draw,inner sep=2pt] (char) {#1};}}

% HYPERLINKS
\usepackage{hyperref}
\hypersetup{
    colorlinks=true,
    urlcolor = black,
    linktoc=all,
    linkcolor=blue,
}

\setmonofont[Scale=0.9]{Fira Code}

% HEADERS
\usepackage{fancyhdr}
    \pagestyle{fancy}
    \lhead{Biometric Matching by Hashing and Applications}
    \rhead{L. Grieder / L. Sidjanski}
    \renewcommand{\footrulewidth}{0.4pt}
    \renewcommand{\headrulewidth}{0.4pt}
\usepackage{etoolbox}  % Define chapter page style
    \patchcmd{\chapter}{\thispagestyle{plain}}{\thispagestyle{fancy}}{}{}


% Capital autoref
\AtBeginDocument{\def\chapterautorefname{Chapitre}}
\AtBeginDocument{\def\sectionautorefname{Section}}
\AtBeginDocument{\def\subsectionautorefname{Sous-Section}}
\AtBeginDocument{\def\figureautorefname{Figure}}

% Custom Commands
\newcommand{\footlink}[2]{\href{#2}{#1}\footnote{#1: \url{#2}}}
\newcommand*{\fullref}[1]{\hyperref[{#1}]{\autoref*{#1} \nameref*{#1}}}

% Counters starts with section number
\counterwithin{figure}{section}
\counterwithin{equation}{section}

\setcounter{tocdepth}{2}

% Colors
\definecolor{myblue}{HTML}{0000ff}
\definecolor{myred}{HTML}{ff0000}
\definecolor{myorange}{HTML}{ff8000}
\definecolor{mygreen}{HTML}{00bf00}

\newcommand{\blue}[1]{{\color{myblue} #1}}
\newcommand{\red}[1]{{\color{myred} #1}}
\newcommand{\orange}[1]{{\color{myorange} #1}}
\newcommand{\green}[1]{{\color{mygreen} #1}}
\newcommand{\black}[1]{{\color{black} #1}}

\newcommand{\definition}[2]{\textbf{#1}: #2}

\begin{document}

% TITLE PAGE
\begin{titlepage}
  \centering
  % EPFL logo
  \includegraphics[width=0.3\linewidth]{latex-img/logo-epfl.png}\\[0.2cm]
  \rule{\textwidth}{0.4pt}\\[2cm]
  % TITLE
  {\LARGE Biometric Matching by Hashing and Applications}\\[2cm]
  % AUTHORS
  {\normalsize Lea Grieder (328216)\\[0.2cm]Leila Sidjanski (330810)}\\[2cm]
  % BACHELOR PROJECT INFO
  {\normalsize Computer Science / Communication Systems\\[0.2cm]EPFL}\\[2cm]
  % DATE
  {\normalsize June 2024}\\[2cm]

  {\normalsize\bfseries Responsible}\\[0.2cm]
  {\normalsize Prof. Serge Vaudenay}\\[2cm]
  \rule{\textwidth}{0.3pt}\\[1cm]
  \includegraphics[width=0.3\linewidth]{latex-img/lasec.jpg}\\[0.2cm]
\end{titlepage}

% ABSTRACT
\newpage
\begin{abstract}
This report explores the application of fuzzy hashing for finger-vein biometric authentication, aiming to enhance security and efficiency in biometric systems. Finger-vein authentication, which leverages internal anatomical features, provides a unique security advantage as these features are less susceptible to replication or theft. The primary focus is on the development and assessment of two algorithms: preHash and postHash. These algorithms transform biometric data into secure hash values that maintain consistency despite slight variations in the input data. The preHash algorithm selects significant bits from the biometric data using a permutation keyed by a secure key, effectively reducing data dimensionality while preserving distinguishing features. The postHash algorithm then compresses these indices into a compact hash. This process is further bolstered by the incorporation of fuzzy extractors, which ensure that even if the stored data is compromised, reconstructing the original biometric data remains computationally infeasible. The experiments conducted involved extensive comparisons, demonstrating the algorithms' efficacy in maintaining a balance between security and performance. Despite some challenges, including the integration of previous work and managing extensive experiment runtimes, the developed system shows promise for secure and efficient biometric authentication.\\

\textbf{Keywords:} Fuzzy hashing, finger-vein biometric authentication, preHash, postHash, fuzzy extractors, biometric security, data compression, biometric hashing.
\end{abstract}


% TABLE OF CONTENTS
\newpage
\tableofcontents
\thispagestyle{fancy}
\pagenumbering{arabic}

% REPORT
\newpage
\section{Introduction}
%\section and \subsection are included in the table of contents

\subsection{Presentation of the Project}
\subsection{Objectives and Scope}
\subsection{Structure of the Report}


% Fuzzy Hashing: Fuzzy hashing is a technique used to generate a hash value that remains consistent even when the input data has minor variations. This is particularly useful in biometrix, when the data captured(like finger-vein patterns) may have slight differences each time due to changes in the environment or the way the biometric trait is presented.

% Purpose of hashing: By storing a hash of the extracted biometric feature rather than the extracted feature itself, the privacy of the user is enhanced. Even if the hash data is compromised, it should not reveal any personal biometric information. Hashed values have fixed sizes which makes storage requirements predictable and efficient. 

% Fuzzy Extractors: takes the concept of fuzzy hashing further by enabling secure error-tolerant biometric authentication. It consists of two main algorithms, Gen (generate) and Rep (reproduce). It enables the secure extraction and reproduction of a key from noisy input data, like biometric data. 

% Registration (Enrollment) process: 

% 1. **Initial Setup:**
%   - Imagine a person named Alex wants to use a secure system that involves finger vein authentication.
%   - Alex chooses a unique identifier, let's say "Alex123" (this is idU), and informs the authority (for example, a bank or a secure facility) about this identifier.

%2. **Biometric Enrollment:**
%   - The authority checks if "Alex123" is available and creates a pseudonym based on it for privacy.
%   - Alex goes to the enrollment center where his finger vein pattern is scanned.
%   - The authority encrypts this biometric data for security and sends it, along with the pseudonym, to a biometric server that stores this information securely.

%3. **Chip Enrollment (when getting a JavaCard):**
%   - For the next step, Alex needs to get a JavaCard that will use his finger vein for authentication.
%   - Alex informs the authority again that he is proceeding with the chip enrollment using the same identifier "Alex123."
%   - The authority captures Alex's finger vein pattern again for verification purposes.
%   - The authority sends this new biometric capture, along with the pseudonym, to the biometric server to match it with the previously stored data.
%   - Once the biometric server confirms the match, it updates its database with Alex's encrypted identifier.
%   - The authority then prepares a JavaCard for Alex by programming it with a new chip identifier, encryption keys, and Alex's encrypted biometric identifier.
%   - This JavaCard information is also sent to a chip server to create a new record linked to Alex.

%In this example, Alex's unique identifier "Alex123" is crucial as it ties all elements of the process together — from biometric enrollment to chip enrollment. It ensures that Alex's biometric data is accurately linked to his JavaCard, allowing him to use finger vein authentication for secure access or transactions. The system maintains privacy by using pseudonyms and encryption, ensuring that Alex's biometric data is protected throughout the process.
\section{Biometric Setting}
%This section is dedicated to establishing the foundational framework for processing and analyzing biometric data derived from finger vein patterns. We explain how the extracted feature vectors from the pipeline are represented and...

This section provides a comprehensive overview of the mathematical and technical aspects involved in fingervein matching, which is essential for comprehending the subsequent discussions in this paper. We will commence by deriving mathematical equations and statements pertinent to our research objectives. Subsequently, we will provide insight into the process of obtaining these equations and statements, evaluating their relevance and implications within the context of our research. 


\subsection{Theoretical Foundations and Concepts}
%biometric template = is the reference model obtained from the finger images, represented as a vector
%Biometric capture = actual data obtained from the finger image during enrollement or authentication, represented as a bitstring X of length n

We begin by delving into the fundamental concepts of biometric data representation. Finger images, designated as biometric templates and formated as \(250\)x\(386\), result in n = \(96'500\) pixels per image. To enhance processing efficiency, these templates are converted into vectors, departing from their original 2-dimension image structure. 

In the biometric context, each finger serves as a biometric subject, with a corresponding biometric capture represented as a bitstring \(X\) of length \(n\). This capture encapsulates the specific vein pixel information extracted from the finger image, while the biometric template serves as a reference mode derived from these images. 

Each of the \(n\) bits of the biometric capture is designated as \(X_1\),..., \(X_n\), with \(X_i\) set to \(1\) if the \(i\)-th bit corresponds to a vein and \(0\) otherwise. In the case where \(i\) and \(X\) are randomly chosen, we define 

\begin{equation} \label{eq:p}
    Pr[X_i = 1] = p \approx 3.6 \% 
\end{equation}

Moving forwards, in biometric authentication and identification, the scoring mechansim plays a crucial role in quantatively determining the similarity between biometric captures. This similarity score holds significance in verifying the identity of an individual (authentication) or identifying potential matches in a database (identification). A higher score indicates a greater ressemblance between captures, while a lower score suggests less similarity. This scoring mechanism is indispensable for ensuring the accuracy and reliability of biometric systems. 

The score of (\(X\), \(Y\)) is computed as

\begin{equation} \label{eq:score}
    \begin{aligned}
        Score(X, Y) &= \frac{HW(X \land Y)}{HW(X) + HW(Y)}\\
        &= \frac{1}{2}-\frac{1}{2}\frac{d_H(X, Y)}{HW(X) + HW(Y)}
    \end{aligned}
\end{equation}

where \(HW\) denotes the Hamming weight and \(d_H\) the Hamming distance. 

In conjunction with the scoring mechanism, Miura matching emerges as a specialized technique for comparing biometric samples. This method entails determining an optimal offset translation, denoted as \(offset * X\) and \(offset * Y\), between two biometric samples to align their features for comparison, thereby compensating for differences in positioning or orientation. The offsets is that maximize the similarity score \(Score(offset_X * X, offset_Y * Y)\) are termed as optimal offsets. Furhtermore, we designate \(\bar{X}\) = \(offset_X * X\) and \(\bar{Y}\) = \(offset_Y * Y\) as the two biometric captures after applying the optimal offset translations \(offset * X\) and \(offset * Y\).
Thus Miura matching and the scoring mechanism are interconnected components, where Miura matching facilitates the computation of the score, thereby offering insights into the similarity between biometric captures and enhancing the reliability of matching algorithms. 





% Start this subsection by introducing the mathematical and theoretical concepts that underpin fuzzy hashing. Discuss the relevance of these concepts in the context of biometric data, focusing on how they enable the creation of reliable and secure hashing mechanisms for inherently noisy data.

%     Key Concepts to Cover:
%         Definition and significance of fuzzy hashing
%         Mathematical principles governing the construction of fuzzy hashes
%         Overview of the biometric setting, including the importance of parameters such as pixel dimensions, vein extraction, and the role of random permutations in hashing

% Subsection 2: Experimental Approach

% In the second subsection, outline the methodology of your experiments designed to test the theoretical underpinnings discussed earlier. Describe the setup, the specific objectives of each experiment, and how these experiments are structured to validate the theoretical models of fuzzy hashing.

%     Key Components to Include:
%         Description of the experimental setup and the data used
%         Explanation of how the experiments are designed to reflect the theoretical aspects of fuzzy hashing
%         Details on the implementation of preHash and postHash functions, and the criteria for their evaluation

% Subsection 3: Verifying Theoretical Predictions

% The final subsection is dedicated to comparing the outcomes of your experiments with the theoretical expectations. This involves analyzing the results, discussing any deviations or confirmations, and what these mean for the validity and reliability of fuzzy hashing in biometric data security.

%     Important Aspects to Discuss:
%         Analysis of experimental results against theoretical predictions
%         Discussion on the accuracy of the fuzzy hashing process, including the matching scores and error rates
%         Implications of the findings for biometric data security and future research directions

% Conclusion of Section 1

% Conclude with a summary of the insights gained from bridging theoretical concepts with empirical evidence. Highlight the importance of this integration for advancing the field of biometric security through fuzzy hashing. Reflect on the potential for future developments and applications stemming from your findings.

\subsection{Part 2}
\subsection{Part 3}
\section{Fuzzy Hashing}
\label{sec:Fuzzy Hashing}
Fuzzy hashing produces consistent cryptographic keys for similar but not identical inputs, enabling recognition of the same biometric trait across different instances despite slight variations. This approach ensures legitimate users are not incorrectly denied access due to minor discrepancies and protects user privacy by storing and using hashed values instead of raw biometric data, making it difficult to reverse-engineer the original data even if unauthorized access occurs. 
In this section, we will delve into the fuzzy hashing algorithm, its corresponding mathematical aspects, conduct experiments, and discuss the results obtained.
\subsection{Algorithm Implementation}

The fuzzy hashing algorithm, known as \textit{preHash}, operates on three inputs: a parameter \(m\), a key \(key\), and an image of finger veins \(X\). It produces a \(m\)-tuple containing the \(m\) smallest indices \(i_j\) such that \(1 <= i_1 < ... < i_m\), and the corresponding pixel \(PRNG_{key}(i_j)\) in \(X\) is identified as vein pixel.

In order to generate the tuple of indices, understanding the mechanics behind its creation is crucial. At the core of this system lies the pseudo-random number generator (PRNG), which operates by producing numbers uniformly and independently. This PRNG mechanism relies on the AES (Advanced Encryption Standard) block cipher functioning in counter (CTR) mode. AES processes data in fixed-size \(128\)-bit blocks. Leveraging AES' crpytographic properties, CTR mode combines a nonce (number only used once) with a counter, encrypting them with a secret key to generate a stream of seemingly random numbers. 
The predictability of this sequence is entirely dictated by the chosen key and nonce. In essence, employing the same key and nonce combination will consistently yield an identical sequence of numbers. 

To generate the nonce, we compute the SHA-256 hash of the key, resulting in a \(256\)-bit hash value. From this hash, we extract a \(128\)-bit value by slicing the first \(16\) bytes. Should a nonce already exist for a given instance, the system seamlessly preserves and reuses it. 

Upon receiving the parameters —key, nonce, and counter— CTR mode produces a \(128\)-bit pseudo-random number. However, to tailor the output to our requirements, we must mask it to \(17\) bits. This adjustment is essential because we aim to generate pseudo-random numbers within the range suitable for an image size, specifically \(96'500\) pixels, which necessitates \(17\) bits for representation. \textcolor{red}{(Faudrait-il écrire cela de manière plus générale? P.ex: produces a 128-bit prng which is then masked to an n bits...?)}

The algorithm, preHash, generates \(m\) smallest indices, denoted by \(i_j\), such that \(j\in{[1, m]} \) and \(1 <= i_1 < ... < i_m\), where each \(i_j\) correponds to an index such that \(X_{PRNG_{key}(i_j)} = 1\). It achieves this by tracking visited indices to maintain uniqueness and avoid infinite loops, and by rigorously verifying that the PRNG stays within the specified bounds (\(< 96'500\) \textcolor{red}{n??}).   

\textcolor{red}{mettre le pseudocode?}

\subsection{Mathematical Concepts}
\textcolor{red}{Not sure how to explain the math, its relevance et comment faire des liens pour passer d'une équations à une autre}

After processing within the pipeline and application of the \textit{preHash} algorithm to the finger vein data, the resulting output consists of indices. 

In the case where the parameters are m = 1, key is random, and k is uniformly distributed random index, we have: 

\begin{equation} \label{eq:preHash1}
    \begin{aligned}
        Pr[preHash_{key}^1(X) = preHash_{key}^1(Y)] &= \sum_{i > 0} Pr[preHash_{key}^1(X)\\
        &= preHash_{key}^1(Y)]\\
        &= \sum_{i > 0} Pr[X_k = Y_k = 0]^{i - 1} Pr[X_k = Y_k = 1]\\
        &= \frac{Pr[X_k = Y_k = 1]}{1 - Pr[X_k = Y_k = 0]}\\
        &= \frac{HW(X \land Y)}{HW(X) + HW(Y) - HW(X \land Y)}\\
        &= \frac{1}{\frac{1}{Score(X, Y)} - 1}
    \end{aligned}
\end{equation}

We notice that there is a direct link with the Miura matching score that we are interested in. The above computation can also be expressed as follows:

\begin{equation} \label{eq:preHash2}
    \begin{aligned}
        Pr[preHash_{key}^1(\bar{X}) = preHash_{key}^1(\bar{Y})] &= \frac{Pr[\bar{X}_k = \bar{Y}_k = 1]}{1 - Pr[\bar{X}_k = \bar{Y}_k = 0]}\\
        &= \frac{\frac{Pr[X_k = 1] + Pr[Y_k = 1]}{2} - \frac{1}{2}Pr[\bar{X}_l \neq \bar{Y}_k]}{\frac{Pr[X_k = 1] + Pr[Y_k = 1]}{2} + \frac{1}{2}Pr[\bar{X}_l \neq \bar{Y}_k]}\\
    \end{aligned}
\end{equation}


Inspired by equations \ref{eq:proba} and \ref{eq:delta}, we make the following approximations:

\begin{equation}
    E\left(\frac{\frac{Pr[X_k = 1] + Pr[Y_k = 1]}{2} - \frac{1}{2}Pr[\bar{X}_l \neq \bar{Y}_k]}{\frac{Pr[X_k = 1] + Pr[Y_k = 1]}{2} + \frac{1}{2}Pr[\bar{X}_l \neq \bar{Y}_k]}\right) \approx \frac{p - \frac{\delta}{2}}{p + \frac{\delta}{2}}
\end{equation}

Hence for (\(X\), \(Y\)) random,

\begin{equation}
    Pr[preHash_{key}^1(offset_X * X) = preHash_{key}^1(offset_Y * Y)] \leq \frac{p - \frac{\delta}{2}}{p + \frac{\delta}{2}}
\end{equation}

where equality is reached for the optimal offset translations. 

Depending on the distribution of (\(X\), \(Y\)), we denote

\begin{equation} \label{eq:mu}
    \mu = \frac{p - \frac{\delta}{2}}{p + \frac{\delta}{2}}
\end{equation}

We have the following figures:

\begin{table}[H]
    \centering
    \renewcommand{\arraystretch}{1.25}\begin{tabular}{|c|c|c|}
        \hline
        $\mu_{same}$ & $\mu_{diff}$ & $\mu_{indep}$\\
        \hline
        $24\%$ & $8.3\%$ & $1.8\%$\\
        \hline
    \end{tabular}
\caption{Comparison of Distributions: $\delta_{same}$, $\delta_{diff}$, and $\delta_{indep}$}
\end{table}

Finally, we have

\begin{equation}
    Pr[preHash_{key}^m(offset_X * X) = preHash_{key}^m(offset_Y * Y)] \leq \mu^m
\end{equation}

where equality is reached for the optimal offset translations.

%This includes determining the upper limits for the probabilities of similarity between different finger veins processed through the same fuzzy hashing parameters. 
\subsection{Part 3}
\section{Compressed Fuzzy Hashing}

\subsection{Data Compression Techniques for m=1, d=4}

{
\renewcommand{\arraystretch}{1.25}
\[
\text{postHash}(i) = \left\{
\begin{array}{lll}
    \text{15} & \text{if } i = 1 & (\text{Pr} = 3.29\%), \\
    \text{14} & \text{if } 2 \leq i \leq 3 & (\text{Pr} = 6.26\%), \\
    \text{13} & \text{if } 4 \leq i \leq 6 & (\text{Pr} = 8.64\%), \\
    \text{12} & \text{if } 7 \leq i \leq 8 & (\text{Pr} = 5.3\%), \\
    \text{11} & \text{if } 9 \leq i \leq 11 & (\text{Pr} = 7.31\%), \\
    \text{10} & \text{if } 12 \leq i \leq 14 & (\text{Pr} = 6.61\%), \\
    \text{9} & \text{if } 15 \leq i \leq 17 & (\text{Pr} = 6\%), \\
    \text{8} & \text{if } 18 \leq i \leq 20 & (\text{Pr} = 5.41\%), \\
    \text{7} & \text{if } 21 \leq i \leq 24 & (\text{Pr} = 6.41\%), \\
    \text{6} & \text{if } 25 \leq i \leq 29 & (\text{Pr} = 6.9\%), \\
    \text{5} & \text{if } 30 \leq i \leq 34 & (\text{Pr} = 5.83\%), \\
    \text{4} & \text{if } 35 \leq i \leq 41 & (\text{Pr} = 6.7\%), \\
    \text{3} & \text{if } 42 \leq i \leq 50 & (\text{Pr} = 6.6\%), \\
    \text{2} & \text{if } 51 \leq i \leq 62 & (\text{Pr} = 6.2\%), \\
    \text{1} & \text{if } 63 \leq i \leq 82 & (\text{Pr} = 6.13\%), \\
    \text{0} & \text{if } 83 \leq i \leq n & (\text{Pr} = 6.43\%),
\end{array}
\right.
\]
}


Scrambled Domain

\renewcommand{\arraystretch}{1.25}{
\[
\text{postHash}(i) = \left\{
\begin{array}{lll}
    15 & \text{if } i \in \{1\} \cup \{4\} & (\text{Pr} = 6.27\%), \\
    14 & \text{if } i \in \{2, 3\} & (\text{Pr} = 6.26\%), \\
    13 & \text{if } i \in \makecell[tl]{\{5, 6\} \cup \{53\}} & (\text{Pr} = 6.24\%), \\
    12 & \text{if } i \in \makecell[tl]{\{7, 8\} \cup \{38\}} & (\text{Pr} = 6.25\%), \\
    11 & \text{if } i \in \makecell[tl]{\{9, 10\} \cup \{29\}} & (\text{Pr} = 6.25\%), \\
    10 & \text{if } i \in \makecell[tl]{\{12, 13\} \cup \{83\} \cup \{85, \ldots, 92\} \\ \cup \{99\} \cup \{170\}} & (\text{Pr} = 6.24\%), \\
    9  & \text{if } i \in \makecell[tl]{\{15, \ldots, 17\} \cup \{78\}} & (\text{Pr} = 6.25\%), \\
    8  & \text{if } i \in \makecell[tl]{\{18, \ldots, 20\} \cup \{42\}} & (\text{Pr} = 6.24\%), \\
    7  & \text{if } i \in \makecell[tl]{\{21, \ldots, 23\} \cup \{63, 64\} \cup \{71\}} & (\text{Pr} = 6.25\%), \\
    6  & \text{if } i \in \makecell[tl]{\{25, \ldots, 28\} \cup \{61\} \cup \{84\}} & (\text{Pr} = 6.25\%), \\
    5  & \text{if } i \in \makecell[tl]{\{30, \ldots, 34\} \cup \{62\}} & (\text{Pr} = 6.26\%), \\
    4  & \text{if } i \in \makecell[tl]{\{35, \ldots, 37\} \cup \{39, \ldots, 41\} \cup \{57\}} & (\text{Pr} = 6.26\%), \\
    3  & \text{if } i \in \makecell[tl]{\{43, \ldots, 50\} \cup \{59\}} & (\text{Pr} = 6.24\%), \\
    2  & \text{if } i \in \makecell[tl]{\{11\} \cup \{51, 52\} \cup \{54, \ldots, 56\} \\ \cup \{58\} \cup \{60\} \cup \{98\}} & (\text{Pr} = 6.25\%), \\
    1  & \text{if } i \in \makecell[tl]{\{24\} \cup \{65, \ldots, 70\} \cup \{72, \ldots, 77\} \\ \cup \{79, \ldots, 82\}} & (\text{Pr} = 6.25\%), \\
    0  & \text{if } i \in \makecell[tl]{\{14\} \cup \{93, \ldots, 97\} \cup \{101\} \\ \cup \{103, \ldots, 169\} \cup \{171, \ldots, n\}} & (\text{Pr} = 6.24\%)
\end{array}
\right.
\]
}




\subsection{Efficiency Improvement in Hashing Process}
\newpage
\section{Application: Private and Compact Biometric Matching}
\label{Application: Private and Compact Biometric Matching}

This section delves into the practical application of fuzzy hashing within the realm of biometric matching. Employing the Hamming distance for biometric matching offers a systematic approach by iteratively generating \textit{l} iterations of the \textit{PreHash} function, defined as:

\begin{equation}
    \begin{aligned}
        Hash_{\text{key}}^m &= Hash_{\text{key}_1, \ldots, \text{key}_l}^m(X)\\
        &= (PreHash_{\text{key}_1}^m(X), \ldots, PreHash_{\text{key}_l}^m(X))
    \end{aligned}
\end{equation}

Subsequently, the Hamming distance between the resulting hash values of two biometric samples \(X\) and \(Y\) is calculated as:

\[d_H(Hash_{key}(X), Hash_{key}(Y)) = \# \{i: PreHash_{key_i}(X) \neq PreHash_{key_i}(Y)\}\]

This expression quantifies the instances "i" where the outputs of the \textit{PreHash} function differ between samples \(X\) and \(Y\).

One notable advantage of this approach is the reduction in size of the stored biometric template. Rather than storing \textit{n} pixels, \textit{ml} integers are stored. Additionally, the key renders the reference \textit{PreHash} less privacy-sensitive compared to a biometric template. Specifically, if the key is known, each integer in the hash discloses about \(\frac{1}{p}\) pixels, revealing \(\frac{ml}{p}\) pixels at worst.

Similarly, when employing the Hamming distance for biometric matching through the iterative generation of \textit{l} iterations of the \textit{PostHash} function, analogous advantages arise. Here, the stored biometric template is condensed to \textit{mld} integers instead of \textit{n} pixels. Furthermore, the key diminishes the sensitivity of the reference \textit{PostHash} in terms of privacy, exposing \(\frac{mld}{p}\) pixels at most if known. Additionally, \textit{PostHash} contributes to leakage reduction.

It's crucial to note that for both scenarios, additional privacy safeguards can be implemented, for instance a restricted access to the key. Hence, the intricacies of the biometric infrastructure must be addressed on a case-by-case basis.

\subsection{Theoretical Foundations of FPR and FNR within Fuzzy Hashing Systems}

The transformation of biometric data into a hash, whether through the \textit{PreHash} method or its compressed counterpart, \textit{PostHash}, is significant in shaping the system's operational efficiency and security. The inherent privacy preservation achieved by either methodology significantly impacts the system's performance metrics, notably the \hyperref[def:FNR]{False Negative Rate (FNR)} and \hyperref[def:FPR]{False Positive Rate (FPR)}. By encapsulating biometric information into a condensed form, the system not only optimizes storage but also diminishes the potential for unauthorized access to sensitive data. This strategic conversion process not only ensures data integrity but also fosters a robust defense against potential security breaches. Furthermore, the judicious selection of hash generation techniques bolsters the system's discrimination capabilities, thereby minimizing the occurrence of false rejections and acceptances, consequently enhancing the overall accuracy and reliability of biometric matching.

We establish a threshold \(t\) to evaluate the match between two biometric samples, \(X\) and \(Y\), by analyzing the Hamming distance between their hash values. A match is confirmed if the difference between \(l\) (the total iterations) and the Hamming distance is equal to or exceeds the threshold \(t\), expressed as: \[l - d_H(Hash_{\text{key}}(X), Hash_{\text{key}}(Y)) \geq t\]
By leveraging the approximation to a normal distribution, we establish the False Negative Rate (FNR) as:

\[FNR = \Phi\left( \frac{t - l\mu_{\text{same}}^m}{\sqrt{l\mu_{\text{same}}^m}} \right)\]

Here, \(\Phi\) denotes the Cumulative Distribution Function (CDF) of \(\mathcal{N}(0, 1)\).

In contrast, we define the False Positive Rate (FPR) as:

\[FPR = \Phi\left( \frac{t - l\mu_{\text{diff}}^m}{\sqrt{l\mu_{\text{diff}}^m}} \right)\]

These formulations allow for the evaluation of false match rates based on the standard deviation and mean of the distributions for same and different samples, respectively. The upper bound \(\mu_{same}\) and \(\mu_{diff}\) are defined in Equation \ref{eq:mu}.

For instance, employing \(\Phi(-2.33) \approx 1\%\) as a benchmark, we calculate the threshold (\(t\)) from parameters \(m\) and \(l\) to achieve an FNR of 1\% and an FPR \(\leq 2^{-36} \). The resulting set of parameters is as follows: 

\begin{table}[htbp] 
    \centering
    \begin{tabular}{|c|c|c|c|c|c|c|}
        \hline
        \textit{m} & \textit{l} & \textit{t} & \textit{l}\(\mu_{\text{same}}^m\) & \textit{l}\(\mu_{\text{diff}}^m\) & \textit{FNR} & \textit{FPR} \\
        \hline
        1 & 637 & 118 & 146.6 & 49.2 & 1\% & \(2^{-37}\) \\
        2 & 961 & 34 & 50.0 & 5.7 & 1\% & \(2^{-38}\) \\
        3 & 2569 & 18 & 31.3 & 1.2 & 1\% & \(2^{-41}\) \\
        4 & 8481 & 12 & 23.8 & 0.3 & 1\% & \(2^{-43}\) \\
        5 & 32 999 & 11 & 21.3 & 0.1 & 1\% & \(2^{-51}\) \\
        6 & 140 090 & 10 & 20.8 & 0.0 & 1\% & \(2^{-67}\) \\
        7 & 568 315 & 9 & 19.5 & 0.0 & 1\% & \(2^{-51}\) \\
        8 & 2 841 573 & 11 & 22.4 & 0.0 & 1\% & \(2^{-120}\) \\
        \hline
    \end{tabular}
    \caption{Parameterization Results for FNR and FPR Calculation}
    \label{tab:parameterization}
\end{table}


\subsection{Experimental Derivation of the FPR and FNR for m=1 and d=4 (?)}
\section{Conclusion}

\subsection{Challenges and Future Directions}

\section{Definitions}
\begin{description}
    \item[Equal Error Rate (EER)] \label{def:EER} Metric used to evaluate the performance of a system. It represents the point at which the system's \hyperref[def:FAR]{false acceptance rate (FAR)} equals its \hyperref[def:FRR]{false rejection rate (FRR)}. A lower EER indicates a more accurate and reliable system as it signifies a balanced trade-off between security (minimizing FAR) and usability (minimizing FRR)

    \item[False Positive Rate (FPR)] \label{def:FPR} This is the probability of incorrectly accepting an unauthorized user

    \item[False Negative Rate (FNR)] \label{def:FNR} This is the the probability of incorrectly rejecting an authorized user

    \item[Hash Function] \label{def:Hash_Function} A hash function is an algorithm that converts input data of any size to a smaller fixed-size string of characters, which typically acts as a data fingerprint. The output, known as a hash, is unique for different inputs in ideal cases, making hash functions crucial for cryptography, data integrity, and indexing in databases.

    \item[Fuzzy Extractors] \label{def:Fuzzy_Extractors} Fuzzy extractors are cryptographic tools designed to reliably and securely generate a consistent, reproducible cryptographic key from biometric data or other noisy inputs that are inherently inconsistent. They enable the extraction of a stable key from an input that may vary slightly over different measurements, ensuring that even with minor variations, the same key can be reliably regenerated. This process typically involves two main components: a generator that produces a stable key and some public data from an initial input, and a reproducer that can regenerate the original key from a similar but not identical input using the public data.

    \item[Uniform Distribution] \label{def:Uniform Distribution} A uniform distribution signifies that each outcome within a set has an equal chance of occurring. In the context of finger vein patterns, it means any bit \(i\) in the biometric capture, representing either the presence or absence of a vein, is equally likely to be selected for analysis.

    \item[Hamming Distance] \label{def:Hamming Distance} The number of positions at which two biometric strings of equal length differ. It measures the similarity between the strings, with a lower distance indicating higher similarity.

    \item[Hamming Weight] \label{def:Hamming Weight} The number of 1's in a biometric string, indicating the presence of veins.

    \item[AES in CTR mode] \label{def:AES CTR mode} AES in Counter (CTR) mode is an encryption method that transforms a block cipher into a stream cipher. It achieves this by encrypting sequential counter values using AES. Each counter value is unique for each block of data, typically starting from an initial nonce (number used once) and incremented for subsequent blocks. This encryption method is traditionally used to encrypt data by generating a sequence of encrypted counters, which are then XORed with plaintext to produce ciphertext, allowing for encryption of arbitrary-sized data without padding. However, in the context of fuzzy hashing for biometric matching, AES-CTR is repurposed to generate a pseudorandom sequence, not for encrypting data but for selecting specific indices from a biometric template. This application leverages AES-CTR's cryptographic strength to ensure unpredictability and determinism in the selection process.

    \item[SHA-256 Hash] \label{def:SHA-256} SHA-256 is a cryptographic hash function in the SHA-2 family that produces a fixed-size 256-bit (32-byte) hash value from an input of arbitrary length. It has three fundamental properties: pre-image resistance, making it computationally infeasible to reverse the hash to find the original input; second pre-image resistance, preventing the discovery of a different input producing the same hash as a given input; and collision resistance, making it highly unlikely to find two different inputs that yield the same hash output. These properties ensure data integrity and security across various digital applications.

\end{description}


\newpage
\printbibliography[heading=bibintoc, title={References}]

\end{document}

