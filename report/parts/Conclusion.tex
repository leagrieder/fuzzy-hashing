\newpage
\newpage
\section{Application: Private and Compact Biometric Matching}
\label{Application: Private and Compact Biometric Matching}

This section delves into the practical application of fuzzy hashing within the realm of biometric matching. Employing the Hamming distance for biometric matching offers a systematic approach by iteratively generating \textit{l} iterations of the \textit{PreHash} function, defined as:

\begin{equation}
    \begin{aligned}
        Hash_{\text{key}}^m &= Hash_{\text{key}_1, \ldots, \text{key}_l}^m(X)\\
        &= (PreHash_{\text{key}_1}^m(X), \ldots, PreHash_{\text{key}_l}^m(X))
    \end{aligned}
\end{equation}

Subsequently, the Hamming distance between the resulting hash values of two biometric samples \(X\) and \(Y\) is calculated as:

\[d_H(Hash_{key}(X), Hash_{key}(Y)) = \# \{i: PreHash_{key_i}(X) \neq PreHash_{key_i}(Y)\}\]

This expression quantifies the instances "i" where the outputs of the \textit{PreHash} function differ between samples \(X\) and \(Y\).

One notable advantage of this approach is the reduction in size of the stored biometric template. Rather than storing \textit{n} pixels, \textit{ml} integers are stored. Additionally, the key renders the reference \textit{PreHash} less privacy-sensitive compared to a biometric template. Specifically, if the key is known, each integer in the hash discloses about \(\frac{1}{p}\) pixels, revealing \(\frac{ml}{p}\) pixels at worst.

Similarly, when employing the Hamming distance for biometric matching through the iterative generation of \textit{l} iterations of the \textit{PostHash} function, analogous advantages arise. Here, the stored biometric template is condensed to \textit{mld} integers instead of \textit{n} pixels. Furthermore, the key diminishes the sensitivity of the reference \textit{PostHash} in terms of privacy, exposing \(\frac{mld}{p}\) pixels at most if known. Additionally, \textit{PostHash} contributes to leakage reduction.

It's crucial to note that for both scenarios, additional privacy safeguards can be implemented, for instance a restricted access to the key. Hence, the intricacies of the biometric infrastructure must be addressed on a case-by-case basis.

\subsection{Theoretical Foundations of FPR and FNR within Fuzzy Hashing Systems}

The transformation of biometric data into a hash, whether through the \textit{PreHash} method or its compressed counterpart, \textit{PostHash}, is significant in shaping the system's operational efficiency and security. The inherent privacy preservation achieved by either methodology significantly impacts the system's performance metrics, notably the \hyperref[def:FNR]{False Negative Rate (FNR)} and \hyperref[def:FPR]{False Positive Rate (FPR)}. By encapsulating biometric information into a condensed form, the system not only optimizes storage but also diminishes the potential for unauthorized access to sensitive data. This strategic conversion process not only ensures data integrity but also fosters a robust defense against potential security breaches. Furthermore, the judicious selection of hash generation techniques bolsters the system's discrimination capabilities, thereby minimizing the occurrence of false rejections and acceptances, consequently enhancing the overall accuracy and reliability of biometric matching.

We establish a threshold \(t\) to evaluate the match between two biometric samples, \(X\) and \(Y\), by analyzing the Hamming distance between their hash values. A match is confirmed if the difference between \(l\) (the total iterations) and the Hamming distance is equal to or exceeds the threshold \(t\), expressed as: \[l - d_H(Hash_{\text{key}}(X), Hash_{\text{key}}(Y)) \geq t\]
By leveraging the approximation to a normal distribution, we establish the False Negative Rate (FNR) as:

\[FNR = \Phi\left( \frac{t - l\mu_{\text{same}}^m}{\sqrt{l\mu_{\text{same}}^m}} \right)\]

Here, \(\Phi\) denotes the Cumulative Distribution Function (CDF) of \(\mathcal{N}(0, 1)\).

In contrast, we define the False Positive Rate (FPR) as:

\[FPR = \Phi\left( \frac{t - l\mu_{\text{diff}}^m}{\sqrt{l\mu_{\text{diff}}^m}} \right)\]

These formulations allow for the evaluation of false match rates based on the standard deviation and mean of the distributions for same and different samples, respectively. The upper bound \(\mu_{same}\) and \(\mu_{diff}\) are defined in Equation \ref{eq:mu}.

For instance, employing \(\Phi(-2.33) \approx 1\%\) as a benchmark, we calculate the threshold (\(t\)) from parameters \(m\) and \(l\) to achieve an FNR of 1\% and an FPR \(\leq 2^{-36} \). The resulting set of parameters is as follows: 

\begin{table}[htbp] 
    \centering
    \begin{tabular}{|c|c|c|c|c|c|c|}
        \hline
        \textit{m} & \textit{l} & \textit{t} & \textit{l}\(\mu_{\text{same}}^m\) & \textit{l}\(\mu_{\text{diff}}^m\) & \textit{FNR} & \textit{FPR} \\
        \hline
        1 & 637 & 118 & 146.6 & 49.2 & 1\% & \(2^{-37}\) \\
        2 & 961 & 34 & 50.0 & 5.7 & 1\% & \(2^{-38}\) \\
        3 & 2569 & 18 & 31.3 & 1.2 & 1\% & \(2^{-41}\) \\
        4 & 8481 & 12 & 23.8 & 0.3 & 1\% & \(2^{-43}\) \\
        5 & 32 999 & 11 & 21.3 & 0.1 & 1\% & \(2^{-51}\) \\
        6 & 140 090 & 10 & 20.8 & 0.0 & 1\% & \(2^{-67}\) \\
        7 & 568 315 & 9 & 19.5 & 0.0 & 1\% & \(2^{-51}\) \\
        8 & 2 841 573 & 11 & 22.4 & 0.0 & 1\% & \(2^{-120}\) \\
        \hline
    \end{tabular}
    \caption{Parameterization Results for FNR and FPR Calculation}
    \label{tab:parameterization}
\end{table}


\subsection{Experimental Derivation of the FPR and FNR for m=1 and d=4 (?)}
\section{Conclusion}
In this report, we have explored the application of fuzzy hashing to finger-vein biometric authentication, aiming to enhance both security and efficiency. Our investigation included the development and assessment of both the preHash and postHash algorithms, which transform biometric data into secure hash values that maintain consistency despite slight variations in the input data. Got it. The results of the experiments conducted throughout our project generally aligned well with the predicted values. However, as detailed in Section~\ref{sec:Application: Private and Compact Biometric Matching}, some discrepancies arose because certain assumptions of independence may not have been accurate. To address this, we modified our preHash algorithm to ensure that no vein pixel was repeated across the \(l\) iterations of preHash, and performed the first experiment from Table~\ref{tab:theoretical_parameterization_PreHash} using this new algorithm implementation. Despite these efforts, the results did not improve. This could be due to potential errors in the algorithmic implementation, or it may indicate that the chosen approach was not optimal.


\subsection{Challenges and Future Directions}
Throughout our work on this project, we encountered numerous challenges. The first significant difficulty was integrating and building upon the work of previous students. Understanding previously written code can be extremely challenging, highlighting the crucial importance of thorough documentation. We wrote a substantial amount of code for our project, ranging from algorithms such as preHash and postHash to various experimental procedures. Given the initial struggle to comprehend the existing codebase, we prioritized documenting our contributions meticulously. This included adding detailed comments and enhancing the README file to ensure clarity and ease of understanding for future developers. We believe that the comprehensive documentation we have provided will significantly benefit anyone who continues to work on this project, ensuring that our code is accessible and easy to understand.

Another significant challenge was running the experiments. With a dataset of 800 images, each experiment generated four populations of results:

- Population I: Comparisons between same biometric subjects taken from camera 1 (\(1'591\) comparisons).\\
- Population III: Comparisons between same biometric subjects taken from camera 2 (\(1'591\) comparisons).\\
- Population II: Comparisons between different biometric subjects taken from camera 1 (\(37'809\) comparisons).\\
- Population IV: Comparisons between different biometric subjects taken from camera 2 (\(37'809\) comparisons).

This amounted to a total of 78,800 comparisons per experiment, making the runtime extremely long and we spent alot of time waiting for results. We should have addressed this issue earlier but only realized the severity of the runtime problem when conducting experiments for Section~\ref{Application: Private and Compact Biometric Matching}. The extended runtime for multiple iterations of preHash and postHash combinations for each image highlighted the inefficiency. To tackle this, we revisited the previously developed pipeline code and implemented parallel processing. Additionally, we gained access to more powerful ressources (SCITAS\cite{ref4}) to run our code more efficiently. These changes drastically reduced the runtime: experiments that previously took over 24 hours to complete were reduced to just 1 hour.

However, it's important to note that despite the improved runtime, the experiments still consume a lot of resources. Some experiments, such as the last few rows of Table~\ref{tab:theoretical_parameterization_PreHash}, still take between 1 and 4 days to run. This issue can only be further addressed by refining the actual pipeline, specifically the extraction of feature vectors, and by converting the code, which is currently written in Python, to a lower-level language. This transition would likely result in more efficient code execution and reduced resource consumption. The current system requires this especially because future work will be working on a concept called "1:N matching", which adds another layer of complexity to the system, as this process involves comparing a single biometirc input against a database of multiple potential matches to identify the closest match or matches accurately.