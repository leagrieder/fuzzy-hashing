\section{Fuzzy Hashing}
Fuzzy hashing produces consistent cryptographic keys for similar but not identical inputs, enabling recognition of the same biometric trait across different instances despite slight variations. This approach ensures legitimate users are not incorrectly denied access due to minor discrepancies and protects user privacy by storing and using hashed values instead of raw biometric data, making it difficult to reverse-engineer the original data even if unauthorized access occurs. 
In this section, we will delve into the fuzzy hashing algorithm, its corresponding mathematical aspects, conduct experiments, and discuss the results obtained.
\subsection{Algorithm implementation}

%The fuzzy hashing algorithm, named preHash, requires three arguments: a parameter $m$, a key $key$, and an image of finger veins $X$. It produces a tuple containing the $m$ first indices $i_j$ such that $1 <= i_1 < ... < i_m$, and $X_{PRNG_{key}(i_j)} = 1$. In other words, $i_1$ represents the smallest index where the pixel at $PRNG_{key}(i_1)$ in image $X$ is a vein pixel. Subsequently, $i_2$ is the smallest index larger than $i_1$, and so on. 

The fuzzy hashing algorithm, known as preHash, operates on three inputs: a parameter $m$, a key $key$, and an image of finger veins $X$. It produces a tuple containing the $m$ smallest indices $i_j$ such that $1 <= i_1 < ... < i_m$, and the corresponding pixels in $X$ are identified as vein pixels.

These indices are calculated using a pseudorandom number generator, denoted as $PRNG$. Such generators produce a deterministic sequence of numbers based on a random seed, represented by the variable $key$. The sequence is entirely determined by this initial seed. In our context, we assume a $PRNG_{key}$ generating a sequence of pseudorandom elements from the set \{1, . . . , n\} based on the seed $key$. To simplify notation, we denote the ith element generated by $PRNG_{key}$ as $PRNG_{key}(i)$.

\subsection{Mathematical concepts}

After processing within the pipeline and application of the preHash algorithm to the finger vein data, the resulting output consists of indices. Each index within the resulting output corresponds to a pixel in X, with the pixel indexed by PRNG$_{key}$(i) representing a vein pixel. We aim to assess the probability of these indices being identical across outputs from different finger veins when utilizing the same parameters. This results interests us, because we would like to ??


In the case where the parameters are m = 1 and key is random, we have: 
BORDEL JE NARRIVE PAS A ECRIRE EN LATEX MATH I GIVE UP
%$Pr[preHash_{key}^1(\bar{X}) = preHash_{key}^1(\bar{Y})] =  \sum_{i > 0} Pr[preHash_{key}^1(\bar{X}) = preHash_{key}^1(\bar{Y})] = \sum_{i > 0} Pr[X_k = Y_k = 0]^{i - 1} Pr[X_k = Y_k = 1] = \frac{Pr[X_k = Y_k = 1]}{1 - Pr[X_k = Y_k = 0]} = \frac{HW(X \land Y)}$





%This includes determining the upper limits for the probabilities of similarity between different finger veins processed through the same fuzzy hashing parameters. 
\subsection{Part 3}