\newpage
\section{Fuzzy Hashing}
\label{sec:Fuzzy Hashing}
Fuzzy hashing produces consistent cryptographic keys for similar but not identical inputs, enabling recognition of the same biometric trait across different instances despite slight variations. This approach ensures legitimate users are not incorrectly denied access due to minor discrepancies and protects user privacy by storing and using hashed values instead of raw biometric data, making it difficult to reverse-engineer the original data even if unauthorized access occurs. 
In this section, we will delve into the fuzzy hashing algorithm, its corresponding mathematical aspects, conduct experiments, and discuss the results obtained.
\subsection{Algorithm Implementation}

The fuzzy hashing algorithm, known as \textit{preHash}, operates on three inputs: a parameter \(m\), a key \(key\), and an image of finger veins \(X\). It produces a \(m\)-tuple containing the \(m\) smallest indices \(i_j\) such that \(1 <= i_1 < ... < i_m\), and the corresponding pixel \(PRNG_{key}(i_j)\) in \(X\) is identified as vein pixel.

In order to generate the tuple of indices, understanding the mechanics behind its creation is crucial. At the core of this system lies the pseudo-random number generator (PRNG), which operates by producing numbers uniformly and independently. This PRNG mechanism relies on the AES (Advanced Encryption Standard) block cipher functioning in counter (CTR) mode. AES processes data in fixed-size \(128\)-bit blocks. Leveraging AES' crpytographic properties, CTR mode combines a nonce (number only used once) with a counter, encrypting them with a secret key to generate a stream of seemingly random numbers. 
The predictability of this sequence is entirely dictated by the chosen key and nonce. In essence, employing the same key and nonce combination will consistently yield an identical sequence of numbers. 

To generate the nonce, we compute the SHA-256 hash of the key, resulting in a \(256\)-bit hash value. From this hash, we extract a \(128\)-bit value by slicing the first \(16\) bytes. Should a nonce already exist for a given instance, the system seamlessly preserves and reuses it. 

Upon receiving the parameters —key, nonce, and counter— CTR mode produces a \(128\)-bit pseudo-random number. However, to tailor the output to our requirements, we must mask it to \(17\) bits. This adjustment is essential because we aim to generate pseudo-random numbers within the range suitable for an image size, specifically \(96'500\) pixels, which necessitates \(17\) bits for representation. \textcolor{red}{(Faudrait-il écrire cela de manière plus générale? P.ex: produces a 128-bit prng which is then masked to an n bits...?)}

The algorithm, preHash, generates \(m\) smallest indices, denoted by \(i_j\), such that \(j\in{[1, m]} \) and \(1 <= i_1 < ... < i_m\), where each \(i_j\) correponds to an index such that \(X_{PRNG_{key}(i_j)} = 1\). It achieves this by tracking visited indices to maintain uniqueness and avoid infinite loops, and by rigorously verifying that the PRNG stays within the specified bounds (\(< 96'500\) \textcolor{red}{n??}).   

\textcolor{red}{mettre le pseudocode?}

\subsection{Mathematical Concepts}
\textcolor{red}{Not sure how to explain the math, its relevance et comment faire des liens pour passer d'une équations à une autre}

After processing within the pipeline and application of the \textit{preHash} algorithm to the finger vein data, the resulting output consists of indices. 

In the case where the parameters are m = 1, key is random, and k is uniformly distributed random index, we have: 

\begin{equation} \label{eq:preHash1}
    \begin{aligned}
        Pr[preHash_{key}^1(X) = preHash_{key}^1(Y)] &= \sum_{i > 0} Pr[preHash_{key}^1(X)\\
        &= preHash_{key}^1(Y)]\\
        &= \sum_{i > 0} Pr[X_k = Y_k = 0]^{i - 1} Pr[X_k = Y_k = 1]\\
        &= \frac{Pr[X_k = Y_k = 1]}{1 - Pr[X_k = Y_k = 0]}\\
        &= \frac{HW(X \land Y)}{HW(X) + HW(Y) - HW(X \land Y)}\\
        &= \frac{1}{\frac{1}{Score(X, Y)} - 1}
    \end{aligned}
\end{equation}

We notice that there is a direct link with the Miura matching score that we are interested in. The above computation can also be expressed as follows:

\begin{equation} \label{eq:preHash2}
    \begin{aligned}
        Pr[preHash_{key}^1(\bar{X}) = preHash_{key}^1(\bar{Y})] &= \frac{Pr[\bar{X}_k = \bar{Y}_k = 1]}{1 - Pr[\bar{X}_k = \bar{Y}_k = 0]}\\
        &= \frac{\frac{Pr[X_k = 1] + Pr[Y_k = 1]}{2} - \frac{1}{2}Pr[\bar{X}_l \neq \bar{Y}_k]}{\frac{Pr[X_k = 1] + Pr[Y_k = 1]}{2} + \frac{1}{2}Pr[\bar{X}_l \neq \bar{Y}_k]}\\
    \end{aligned}
\end{equation}


Inspired by equations \ref{eq:proba} and \ref{eq:delta}, we make the following approximations:

\begin{equation}
    E\left(\frac{\frac{Pr[X_k = 1] + Pr[Y_k = 1]}{2} - \frac{1}{2}Pr[\bar{X}_l \neq \bar{Y}_k]}{\frac{Pr[X_k = 1] + Pr[Y_k = 1]}{2} + \frac{1}{2}Pr[\bar{X}_l \neq \bar{Y}_k]}\right) \approx \frac{p - \frac{\delta}{2}}{p + \frac{\delta}{2}}
\end{equation}

Hence for (\(X\), \(Y\)) random,

\begin{equation}
    Pr[preHash_{key}^1(offset_X * X) = preHash_{key}^1(offset_Y * Y)] \leq \frac{p - \frac{\delta}{2}}{p + \frac{\delta}{2}}
\end{equation}

where equality is reached for the optimal offset translations. 

Depending on the distribution of (\(X\), \(Y\)), we denote

\begin{equation} \label{eq:mu}
    \mu = \frac{p - \frac{\delta}{2}}{p + \frac{\delta}{2}}
\end{equation}

We have the following figures:

\begin{table}[H]
    \centering
    \renewcommand{\arraystretch}{1.25}\begin{tabular}{|c|c|c|}
        \hline
        $\mu_{same}$ & $\mu_{diff}$ & $\mu_{indep}$\\
        \hline
        $24\%$ & $8.3\%$ & $1.8\%$\\
        \hline
    \end{tabular}
\caption{Comparison of Distributions: $\delta_{same}$, $\delta_{diff}$, and $\delta_{indep}$}
\end{table}

Finally, we have

\begin{equation}
    Pr[preHash_{key}^m(offset_X * X) = preHash_{key}^m(offset_Y * Y)] \leq \mu^m
\end{equation}

where equality is reached for the optimal offset translations.

%This includes determining the upper limits for the probabilities of similarity between different finger veins processed through the same fuzzy hashing parameters. 
\subsection{Part 3}

\section{Compressed Fuzzy Hashing}
\label{sec:Compressed Fuzzy Hashing}

Compressed fuzzy hashing generates concise hashes of files or data, capturing key features while allowing for slight variations - a concept referred to as "fuzziness". Unlike traditional cryptographic hashing methods, which produce fixed-length hashes vulnerable to significant changes even with minor input alterations, compressed fuzzy hashing offers greater resilience. Even when alterations occur, the hashing algorithm can still identify similar content. 

This section delves into the implementation of the compressed fuzzy hashing algorithm, known as \textit{PostHash}, its underlying mathematical principles, and present certain experimental results.

\subsection{PostHashing Algorithm}

The \textit{PostHash} algorithm, constituting the second step in the fuzzy hashing process, generates compact hashes based on the indices derived from the \textit{PreHash} algorithm (see section \ref{sec:Fuzzy Hashing}). It maps a list of indices to a hash \(h_1|| \ldots || h_m\), where each \(h_i\) \(i \in [1, m]\), is an integer in the range \([0, \ldots, D-1]\). Here, \(D = 2^d\), with d indicating the number of bits used to represent a single index. 

Algorithm Inputs and Outputs:
\begin{enumerate}
    \item \textbf{Inputs}: As illustrated in Figure~\ref{postHash Algorithm}, the algorithm takes as input:
    \begin{itemize}
        \item \textbf{A tuple of indices}: A tuple of indices \((i_1, \ldots, i_m)\), which is the output of the \textit{PreHash} algorithm
    \end{itemize}
    \item \textbf{Output}: Converted indices, forming a compact hash \(h_1|| \ldots || h_m\), where \(h_i\) \(i \in [1, m]\), is an integer in \([0, \ldots, D-1]\)
\end{enumerate} 

Detailed Process of \textit{PostHash}:
\begin{itemize}
    \item \textbf{Subroutine}: For each index produced by the \textit{PreHash} algorithm, \textit{PostHash} employs a subroutine \(T\) (see Figure ~\ref{Subroutine Algorithm}). This subroutine returns 0 if the input is out of range and performs a table lookup otherwise. It maps an index \(i\) to an integer nearly uniformly distributed between \([0, D-1]\). The implementation is as follows:
    \begin{itemize}
        \item \textbf{Inputs}: 
        \begin{itemize}
            \item \textbf{Index}: The index in the indices array that we wish to convert
            \item \textbf{d}: Bitlength of the desired hash index, indicating the number of bits used to represent a single index \(h_i\) \(i \in [1, m]\). The relationship \(D = 2^d\) expresses the total number of possible hash indices that can be represented with d bits
        \end{itemize}
        \item \textbf{Output}: \(h_i\), the mapped integer
    \end{itemize}
\end{itemize}

The \textit{postHash} algorithm generates integers \(h_1 || \ldots || h_m\), where each \(h_i \in [0, D-1]\). Here, \(D = 2^d\), with \(d\) indicating the number of bits used to represent a single index. 

\begin{algorithm}
    \begin{algorithmic}[1]
    \caption{\textit{postHash} Algorithm}
    \label{postHash Algorithm}
    \Function{(postHash $\circ$ preHash$_\text{key}^m$)}{$X$}
    \State $\text{hash} = []$
    \State $i' \gets 0$
    \For{$i \in \text{indices}$}
        \State $h_i \gets \text{Subroutine}(i - i', d)$
        \State $\text{hash.append}(h_i)$
        \State $i' \gets i$
    \EndFor 
    \State \Return{$h_1, \ldots, h_m$}
    \EndFunction
    \end{algorithmic}
    \end{algorithm}
    
    \begin{algorithm}
    \begin{algorithmic}[1]
    \caption{\textit{Subroutine} Algorithm}
    \label{Subroutine Algorithm}
    \Function{\text{Subroutine}}{$i$, $d$}
    \State $p = 0.0329$
    \State $h_i = \lfloor 2^d (1-p)^{i} \rfloor$
    \State \Return{$h_i$}
    \EndFunction
    \end{algorithmic}
    \end{algorithm}

\subsection{Assessing Similarity of Biometric Inputs After PostHash Application}
\label{sec:q}

After processing the finger images through the pipeline outlined in Pipeline~\ref{pipeline_simon} to extract their feature vectors, and subsequently applying the \textit{postHash} algorithm to the output of \textit{preHash}, the result comprises a set of integers falling within the inclusive range \(h_i \in [0, D-1]\), effectively assigning each index to an integer.

Our methodology assumes that these integers \(h_i\) follow a geometric distribution, with the hash length parameter \(m\) set to \(1\) for generating single-integer hashes. Given that the input for \textit{postHash} is the output of \textit{preHash}, let us denote this relationship as \(Hash_{key}^m(X) = \text{postHash}(\text{preHash}_{key}^m(X))\). We assume the same conditions as discussed in Section~\ref{sec:Fuzzy Hashing}: the key is randomly chosen, and \(k\) represents a uniformly distributed random index. Consequently, the probability of the \textit{Hash} operation yielding the same index for two different inputs \(X\) and \(Y\) can be mathematically characterized as follows:

\[Pr[Hash_{key}^m(d_X) = Hash_{key}^m(d_Y)] \leq \mu^m(1 - \frac{1}{D}) + \frac{1}{D}\]

where equality is reached for the optimal offset translations. 

Depending on the distribution of \(X\), \(Y\) it is denoted

\[q = \mu^m(1 - \frac{1}{D}) + \frac{1}{D}\]

Depending on the values that \(d\) can take on, we computed a few values:
\begin{table}[H]
    \centering
    \renewcommand{\arraystretch}{1.25}\begin{tabular}{|c|c|c|c|}
        \hline
        & $q_{same}$ & $q_{diff}$ & $q_{indep}$\\
        \hline
        m = 1 d = 1 & $60\%$ & $54\%$ & $51\%$\\
        m = 1 d = 2 & $40\%$ & $31\%$ & $26\%$\\
        m = 1 d = 3 & $30\%$ & $20\%$ & $14\%$\\
        m = 1 d = 4 & $25\%$ & $14\%$ & $8\%$\\
        \hline
    \end{tabular}
\caption{Comparison of Distributions: $q_{same}$, $q_{diff}$, and $q_{indep}$}
\end{table}

\subsection{Data Compression Techniques for m=1, d=4}

{
\renewcommand{\arraystretch}{1.25}
\[
\text{postHash}(i) = \left\{
\begin{array}{lll}
    \text{15} & \text{if } i = 1 & (\text{Pr} = 3.29\%), \\
    \text{14} & \text{if } 2 \leq i \leq 3 & (\text{Pr} = 6.26\%), \\
    \text{13} & \text{if } 4 \leq i \leq 6 & (\text{Pr} = 8.64\%), \\
    \text{12} & \text{if } 7 \leq i \leq 8 & (\text{Pr} = 5.3\%), \\
    \text{11} & \text{if } 9 \leq i \leq 11 & (\text{Pr} = 7.31\%), \\
    \text{10} & \text{if } 12 \leq i \leq 14 & (\text{Pr} = 6.61\%), \\
    \text{9} & \text{if } 15 \leq i \leq 17 & (\text{Pr} = 6\%), \\
    \text{8} & \text{if } 18 \leq i \leq 20 & (\text{Pr} = 5.41\%), \\
    \text{7} & \text{if } 21 \leq i \leq 24 & (\text{Pr} = 6.41\%), \\
    \text{6} & \text{if } 25 \leq i \leq 29 & (\text{Pr} = 6.9\%), \\
    \text{5} & \text{if } 30 \leq i \leq 34 & (\text{Pr} = 5.83\%), \\
    \text{4} & \text{if } 35 \leq i \leq 41 & (\text{Pr} = 6.7\%), \\
    \text{3} & \text{if } 42 \leq i \leq 50 & (\text{Pr} = 6.6\%), \\
    \text{2} & \text{if } 51 \leq i \leq 62 & (\text{Pr} = 6.2\%), \\
    \text{1} & \text{if } 63 \leq i \leq 82 & (\text{Pr} = 6.13\%), \\
    \text{0} & \text{if } 83 \leq i \leq n & (\text{Pr} = 6.43\%),
\end{array}
\right.
\]
}


Scrambled Domain

\renewcommand{\arraystretch}{1.25}{
\[
\text{postHash}(i) = \left\{
\begin{array}{lll}
    15 & \text{if } i \in \{1\} \cup \{4\} & (\text{Pr} = 6.27\%), \\
    14 & \text{if } i \in \{2, 3\} & (\text{Pr} = 6.26\%), \\
    13 & \text{if } i \in \makecell[tl]{\{5, 6\} \cup \{53\}} & (\text{Pr} = 6.24\%), \\
    12 & \text{if } i \in \makecell[tl]{\{7, 8\} \cup \{38\}} & (\text{Pr} = 6.25\%), \\
    11 & \text{if } i \in \makecell[tl]{\{9, 10\} \cup \{29\}} & (\text{Pr} = 6.25\%), \\
    10 & \text{if } i \in \makecell[tl]{\{12, 13\} \cup \{83\} \cup \{85, \ldots, 92\} \\ \cup \{99\} \cup \{170\}} & (\text{Pr} = 6.24\%), \\
    9  & \text{if } i \in \makecell[tl]{\{15, \ldots, 17\} \cup \{78\}} & (\text{Pr} = 6.25\%), \\
    8  & \text{if } i \in \makecell[tl]{\{18, \ldots, 20\} \cup \{42\}} & (\text{Pr} = 6.24\%), \\
    7  & \text{if } i \in \makecell[tl]{\{21, \ldots, 23\} \cup \{63, 64\} \cup \{71\}} & (\text{Pr} = 6.25\%), \\
    6  & \text{if } i \in \makecell[tl]{\{25, \ldots, 28\} \cup \{61\} \cup \{84\}} & (\text{Pr} = 6.25\%), \\
    5  & \text{if } i \in \makecell[tl]{\{30, \ldots, 34\} \cup \{62\}} & (\text{Pr} = 6.26\%), \\
    4  & \text{if } i \in \makecell[tl]{\{35, \ldots, 37\} \cup \{39, \ldots, 41\} \cup \{57\}} & (\text{Pr} = 6.26\%), \\
    3  & \text{if } i \in \makecell[tl]{\{43, \ldots, 50\} \cup \{59\}} & (\text{Pr} = 6.24\%), \\
    2  & \text{if } i \in \makecell[tl]{\{11\} \cup \{51, 52\} \cup \{54, \ldots, 56\} \\ \cup \{58\} \cup \{60\} \cup \{98\}} & (\text{Pr} = 6.25\%), \\
    1  & \text{if } i \in \makecell[tl]{\{24\} \cup \{65, \ldots, 70\} \cup \{72, \ldots, 77\} \\ \cup \{79, \ldots, 82\}} & (\text{Pr} = 6.25\%), \\
    0  & \text{if } i \in \makecell[tl]{\{14\} \cup \{93, \ldots, 97\} \cup \{101\} \\ \cup \{103, \ldots, 169\} \cup \{171, \ldots, n\}} & (\text{Pr} = 6.24\%)
\end{array}
\right.
\]
}




\subsection{Efficiency Improvement in Hashing Process}