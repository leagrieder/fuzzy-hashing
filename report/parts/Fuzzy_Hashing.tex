\section{Introduction}
%\section and \subsection are included in the table of contents

\subsection{Presentation of the Project}
\subsection{Objectives and Scope}
\subsection{Structure of the Report}


% Fuzzy Hashing: Fuzzy hashing is a technique used to generate a hash value that remains consistent even when the input data has minor variations. This is particularly useful in biometrix, when the data captured(like finger-vein patterns) may have slight differences each time due to changes in the environment or the way the biometric trait is presented.

% Purpose of hashing: By storing a hash of the extracted biometric feature rather than the extracted feature itself, the privacy of the user is enhanced. Even if the hash data is compromised, it should not reveal any personal biometric information. Hashed values have fixed sizes which makes storage requirements predictable and efficient. 

% Fuzzy Extractors: takes the concept of fuzzy hashing further by enabling secure error-tolerant biometric authentication. It consists of two main algorithms, Gen (generate) and Rep (reproduce). It enables the secure extraction and reproduction of a key from noisy input data, like biometric data. 

% Registration (Enrollment) process: 

% 1. **Initial Setup:**
%   - Imagine a person named Alex wants to use a secure system that involves finger vein authentication.
%   - Alex chooses a unique identifier, let's say "Alex123" (this is idU), and informs the authority (for example, a bank or a secure facility) about this identifier.

%2. **Biometric Enrollment:**
%   - The authority checks if "Alex123" is available and creates a pseudonym based on it for privacy.
%   - Alex goes to the enrollment center where his finger vein pattern is scanned.
%   - The authority encrypts this biometric data for security and sends it, along with the pseudonym, to a biometric server that stores this information securely.

%3. **Chip Enrollment (when getting a JavaCard):**
%   - For the next step, Alex needs to get a JavaCard that will use his finger vein for authentication.
%   - Alex informs the authority again that he is proceeding with the chip enrollment using the same identifier "Alex123."
%   - The authority captures Alex's finger vein pattern again for verification purposes.
%   - The authority sends this new biometric capture, along with the pseudonym, to the biometric server to match it with the previously stored data.
%   - Once the biometric server confirms the match, it updates its database with Alex's encrypted identifier.
%   - The authority then prepares a JavaCard for Alex by programming it with a new chip identifier, encryption keys, and Alex's encrypted biometric identifier.
%   - This JavaCard information is also sent to a chip server to create a new record linked to Alex.

%In this example, Alex's unique identifier "Alex123" is crucial as it ties all elements of the process together — from biometric enrollment to chip enrollment. It ensures that Alex's biometric data is accurately linked to his JavaCard, allowing him to use finger vein authentication for secure access or transactions. The system maintains privacy by using pseudonyms and encryption, ensuring that Alex's biometric data is protected throughout the process.

\section{Fuzzy Hashing}
\label{sec:Fuzzy Hashing}
Fuzzy hashing, as opposed to traditional hashing, produces consistent cryptographic keys for similar but not identical inputs, enabling recognition of the same biometric trait across different instances despite slight variations. This approach ensures legitimate users are not incorrectly denied access due to minor discrepancies and protects user privacy by storing and using hashed values instead of raw biometric data, making it difficult to reverse-engineer the original data even if unauthorized access occurs. 
This section will discuss how we implemented the fuzzy hashing algorithm, its corresponding mathematical aspects and some experiments.

\subsection{PreHashing Algorithm}

The \textit{PreHash} algorithm is the first step in the fuzzy hashing process, designed to manipulate biometric templates extracted from finger vein patterns. It operates on a bitstring \(X\), representing the presence (1) or absence (0) of vein pixels across \(n\) pixels, where \(n=96'500\).

Algorithm Inputs and Outputs:
\begin{enumerate}
    \item \textbf{Inputs}: The algorithm takes three main inputs:
    \begin{itemize}
        \item \textbf{A parameter m}: the number of indices to find
        \item \textbf{A bitstring X}: the feature-extracted vein patterns of a biometric capture
        \item \textbf{A key}: used to initialize a Pseudorandom Number Generator (PRNG)
    \end{itemize}
    \item \textbf{Output}: The algorithm outputs a tuple \((i_1,...,i_m)\) consisting of the \(m\) smallest indices \(i_j\)​ such that \(1 \leq i_1<...<i_m\)​ and the pixel at \(PRNGkey(i_j)\) in \(X\) is identified as a vein pixel. 
\end{enumerate}

Detailed Process of \textit{PreHash}:
\begin{itemize}
    \item \textbf{Initiallization}: Utilizing the provided key, the algorithm initializes a PRNG. This PRNG is based on the \hyperref[def:AES CTR mode]{Advanced Encryption Standard (AES) in Counter (CTR) mode}, ensuring the generation of uniform and independent pseudorandom sequences.

    \item \textbf{Nonce Generation}: A nonce in CTR mode encryption is initialized to zero to maintain simplicity and security. We opted against using a keyed hash function to generate the nonce, as it would tie the nonce to the secret key. Such a dependency would mean that both the nonce and the Pseudo-Random Number Generator (PRNG) would rely on the same key, creating a security risk by concentrating security on a single element. To avoid this, we keep the generation of the nonce separate from the key.

    \item \textbf{Pseudorandom Sequence Generation}: Upon receiving the parameters —key, nonce, and counter— the PRNG utilizes CTR mode to produce a \(128\)-bit pseudo-random number. However, to tailor the output to the project's requirements, it is necessary to mask it to \(17\) bits. This adjustment is essential as it aims to ensure that pseudo-random numbers are generated within a suitable range form image size, specifically \(96'500\) pixels, which requires \(17\) bits for representation\footnote{\(\lceil \log_2(96'500) \rceil = 17\)}. The predictability of this sequence is entirely dictated by the chosen key. In essence, employing the same key will consistently yield an identical nonce and hence an identical sequence of numbers.

    \item \textbf{Selection of Indices}: The algorithm iterates through the generated pseudorandom sequence, selecting the first \(m\) indices corresponding to vein pixels in the biometric template \(X\). This selection process involves a careful mechanism to ensure the uniqueness and proper ordering of indices.

    \item \textbf{Handling the Case \(m >\) (Number of 1's in Vein Image)}: In scenarios where the specified number of vein pixels \(m\) cannot be found due to the absence of sufficient vein pixels within the biometric template, the algorithm incorporates a built-in mechanism to address such situations. Before executing the prehash algorithm, it iterates through the image and verifies if the count of pixels equal to 1 is less than \(m\).

\end{itemize}

\begin{figure}[H]
    \centering
    \includegraphics[width=0.5\linewidth]{latex-img/pseudocode_preHash.png}
    \caption{\textit{preHash} Algorithm}
    \label{preHash Algorithm}
\end{figure}

The algorithm, preHash, illustrated in Figure~\ref{preHash Algorithm} generates \(m\) smallest indices, denoted by \(i_j\), such that \(j\in{[1, m]} \) and \(1 <= i_1 < ... < i_m\), where each \(i_j\) correponds to an index such that \(X_{PRNG_{key}(i_j)} = 1\). It achieves this by rigorously verifying that the numbers produced by PRNG (PRNG[i]) stay within the specified bounds \(\text{PRNG}[i] \in (0, n] \text{ for } i \in [0, m]\).

\subsection{Assessing Similarity of Biometric Inputs After PreHash Application}

After the finger images are processed through the pipeline described in Pipeline~\ref{pipeline_simon} to extract their feature vectors, and the \textit{preHash} algorithm is applied, the outcome is a set of indices that fall within the inclusive range \(\text{PRNG}[i] \in (0, n] \text{ for } i \in [0, m]\), effectively mapping each selected feature to a unique index within the feature vector's length.

In the context of a simplified scenario where the hash length parameter (\(m\)) is set to \(1\), implying the generation of a single-index hash, and assuming a randomly chosen key for the \textit{preHash} algorithm, along with \(k\) representing a uniformly distributed random index, the probability that the \textit{preHash} operation yields the same index for two different inputs \(X\) and \(Y\) can be mathematically delineated as follows:

\begin{equation} \label{eq:preHash1}
    \begin{aligned}
        Pr[preHash_{key}^1(X) = preHash_{key}^1(Y)] &= \sum_{i > 0} Pr[preHash_{key}^1(X)]\\
        &= preHash_{key}^1(Y)\\
        &= \sum_{i > 0} Pr[X_k = Y_k = 0]^{i - 1} Pr[X_k = Y_k = 1]\\
        &= \frac{Pr[X_k = Y_k = 1]}{1 - Pr[X_k = Y_k = 0]}\\
        &= \frac{HW(X \land Y)}{HW(X) + HW(Y) - HW(X \land Y)}\\
        &= \frac{1}{\frac{1}{Score(X, Y)} - 1}
    \end{aligned}
\end{equation}

This equation encapsulates the likelihood of two images, \(X\) and \(Y\), having their singular hash index coincide, based on the presence of matching features identified by the algorithm. The final form of the equation relates the probability to the scoring function between \(X\) and \(Y\), inversely proportional to the score minus one.

It is noticed that there is a direct link with the Miura matching score that is of interest. The direct link between the \textit{preHash} algorithm's outcomes and the Miura matching score lies in their shared foundation of evaluating biometric similarities. Specifically, both methodologies utilize Hamming weight and bitwise operations to assess the overlap between biometric samples, such as finger vein patterns. The \textit{preHash} algorithm, through its probabilistic formula, quantifies the likelihood of matching indices based on feature presence, closely paralleling the Miura score's approach of comparing binary patterns to derive a similarity score. The above computation can also be expressed as follows:

\begin{equation} \label{eq:preHash2}
    \begin{aligned}
        Pr[preHash_{key}^1(\bar{X}) = preHash_{key}^1(\bar{Y})] &= \frac{Pr[\bar{X}_k = \bar{Y}_k = 1]}{1 - Pr[\bar{X}_k = \bar{Y}_k = 0]}\\
        &= \frac{\frac{Pr[X_k = 1] + Pr[Y_k = 1]}{2} - \frac{1}{2}Pr[\bar{X}_l \neq \bar{Y}_k]}{\frac{Pr[X_k = 1] + Pr[Y_k = 1]}{2} + \frac{1}{2}Pr[\bar{X}_l \neq \bar{Y}_k]}\\
    \end{aligned}
\end{equation}

The following approximations are made, inspired by equations p (\ref{eq:proba}) and $\delta$ (\ref{eq:delta}):

\begin{equation}
    E\left(\frac{\frac{Pr[X_k = 1] + Pr[Y_k = 1]}{2} - \frac{1}{2}Pr[\bar{X}_l \neq \bar{Y}_k]}{\frac{Pr[X_k = 1] + Pr[Y_k = 1]}{2} + \frac{1}{2}Pr[\bar{X}_l \neq \bar{Y}_k]}\right) \approx \frac{p - \frac{\delta}{2}}{p + \frac{\delta}{2}}
\end{equation}

The core of this approximation revolves around the expectation formula, which integrates probabilities of feature presence \(Pr[X_k=1]+Pr[Y_k=1]\) and the likelihood of discrepancies between \(X\) and \(Y\), \(Pr[\bar{X}_l \neq \bar{Y}_k]\). This formula essentially aims to quantify the similarity between two biometric samples by considering both the concurrence of features and the instances where they diverge.

Hence for (\(X\), \(Y\)) random,

\begin{equation}
    Pr[preHash_{key}^1(offset_X * X) = preHash_{key}^1(offset_Y * Y)] \leq \frac{p - \frac{\delta}{2}}{p + \frac{\delta}{2}}
\end{equation}

where equality is reached for the optimal offset translations. 

Depending on the distribution of (\(X\), \(Y\)), it is denoted

\begin{equation} \label{eq:mu}
    \mu = \frac{p - \frac{\delta}{2}}{p + \frac{\delta}{2}}
\end{equation}

The following figures are provided:

\begin{table}[H]
    \centering
    \renewcommand{\arraystretch}{1.25}\begin{tabular}{|c|c|c|}
        \hline
        $\mu_{same}$ & $\mu_{diff}$ & $\mu_{indep}$\\
        \hline
        $24\%$ & $8.3\%$ & $1.8\%$\\
        \hline
    \end{tabular}
\caption{Comparison of Distributions: $\delta_{same}$, $\delta_{diff}$, and $\delta_{indep}$}
\end{table}

Finally, it is observed that

\begin{equation}
    Pr[preHash_{key}^m(offset_X * X) = preHash_{key}^m(offset_Y * Y)] \leq \mu^m
\end{equation}

where equality is reached for the optimal offset translations.

%This includes determining the upper limits for the probabilities of similarity between different finger veins processed through the same fuzzy hashing parameters. 
\subsection{Experimental Derivation of the Probabilities \(\mu_{\text{same}}, \mu_{\text{diff}}, \mu_{\text{indep}}\)}

% Start this subsection by introducing the mathematical and theoretical concepts that underpin fuzzy hashing. Discuss the relevance of these concepts in the context of biometric data, focusing on how they enable the creation of reliable and secure hashing mechanisms for inherently noisy data.

%     Key Concepts to Cover:
%         Definition and significance of fuzzy hashing
%         Mathematical principles governing the construction of fuzzy hashes
%         Overview of the biometric setting, including the importance of parameters such as pixel dimensions, vein extraction, and the role of random permutations in hashing

% Subsection 2: Experimental Approach

% In the second subsection, outline the methodology of your experiments designed to test the theoretical underpinnings discussed earlier. Describe the setup, the specific objectives of each experiment, and how these experiments are structured to validate the theoretical models of fuzzy hashing.

%     Key Components to Include:
%         Description of the experimental setup and the data used
%         Explanation of how the experiments are designed to reflect the theoretical aspects of fuzzy hashing
%         Details on the implementation of preHash and postHash functions, and the criteria for their evaluation

% Subsection 3: Verifying Theoretical Predictions

% The final subsection is dedicated to comparing the outcomes of your experiments with the theoretical expectations. This involves analyzing the results, discussing any deviations or confirmations, and what these mean for the validity and reliability of fuzzy hashing in biometric data security.

%     Important Aspects to Discuss:
%         Analysis of experimental results against theoretical predictions
%         Discussion on the accuracy of the fuzzy hashing process, including the matching scores and error rates
%         Implications of the findings for biometric data security and future research directions

% Conclusion of Section 1

% Conclude with a summary of the insights gained from bridging theoretical concepts with empirical evidence. Highlight the importance of this integration for advancing the field of biometric security through fuzzy hashing. Reflect on the potential for future developments and applications stemming from your findings. --> Avons nous vraiment besoin de ça dans cette partie?